\subsection{Layered approach and extensibility}

The purpose of this section is to show that a {\em static} $\to$ {\em dynamic}
and {\em basic} $\to$ {\em complex} approach in designing reflection
can accomodate a wide variety of programming styles and is arguably
the "best" one. We do not propose to add all layers described
below into the standard library. They are mentioned here only to
show that a well designed compile-time reflection is a good foundation
for many (if not all) other reflection facilities.

The Mirror reflection utilities \cite{mirror-doc-cpp11} on which this
proposal is based, implements several distinct components which
are stacked on top of each other. From the low-level metadata, through
a functional-style compile-time interface to a completely dynamic
object-oriented run-time layer (all described in greater detail below).

\subsubsection{Basic metaobjects}
The very basic metadata, which are in Mirror
provided (registered) by the user (or an automated command-line tool) via a set
of preprocessor macros. This approach is both inconvenient and error-prone
in many situations, but also has its advantages.

We propose that a standard compiler should make these metadata available
to the programmer through the static basic metaobject interfaces. These should
serve as the basis for other (standard and non-standard) higher-level
reflection libraries and utilities.

In the Mirror utilities the basic metadata is not used directly by
applications.

\subsubsection{Mirror}

Mirror is a compile-time functional-style reflective programming library,
which is based directly on the basic metadata and is suitable for generic programming,
similar to the standard \verb@type_traits@ library.

Mirror is the original library from which the Mirror reflection utilities started.

It provides a more user-friendly and rich interface than the basic-metaobjects.
and a set of metaprogramming utilities which allow
to write compile-time meta-programs, which can generate efficient
and optimized program code using only those metadata that are required.

The appendix~\ref{appendix-mirror-examples} contains several (rather simple) examples of usage
and the functional style of the algorithms based on metadata provided by Mirror.


\subsubsection{Puddle}

Puddle is a OOP-style (mostly) compile-time interface built on top
of Mirror. It copies the metaobject concept hierarchy of Mirror,
but provides a more "object-ish" interface as shown below:

Instead of Mirror's:

\begin{lstlisting}
static_assert(
  is_public<
    access_type<
      at_c<
        member_variables<
          reflected<person>
        >,
        0
      >
    >
  >::value,
  "Shoot, persons first mem. variable is not public!"
)

\end{lstlisting}

Puddle allows to do the following:

\begin{lstlisting}
assert(
  reflected_type<person>()
    member_variables().
      at_c<0>().
        access_type().
          is_public()
);
\end{lstlisting}

or a more complex example, in which a reflection-based algorithm
traverses the global scope namespace and its nested scopes
and prints information about their members:

\begin{lstlisting}
struct object_printer
{
  std::ostream& out;
  int indent_level;

  std::ostream& indented_output(void)
  {
    for(int i=0;i!=indent_level;++i)
      out << "  ";
    return out;
  }

  template <class MetaObject>
  void print_details(MetaObject obj, mirror::meta_object_tag)
  {
  }

  template <class MetaObject>
  void print_details(MetaObject obj, mirror::meta_scope_tag)
  {
    out << ": ";
    if(obj.members().empty())
    {
      out << "{ }";
    }
    else
    {
      out << "{" << std::endl;
      object_printer print_members = {out, indent_level+1};
      obj.members().for_each(print_members);
      indented_output() << "}";
    }
  }

  template <class MetaObject>
  void print(MetaObject obj, bool last)
  {
    indented_output()
      << obj.self().construct_name()
      << " "
      << obj.base_name();
    print_details(obj, obj.category());
    if(!last) out << ",";
    out << std::endl;
  }
  template <class MetaObject>
  void operator()(MetaObject obj, bool first, bool last)
  {
    print(obj, last);
  }

  template <class MetaObject>
  void operator()(MetaObject obj)
  {
    print(obj, true);
  }


int main(void)
{
  object_printer print = {std::cout, 0};
  print(puddle::adapt<MIRRORED_GLOBAL_SCOPE()>());
  return 0;
}
\end{lstlisting}

which prints the following on the standard output:

\begin{verbatim}
   global scope : {
     namespace std: {
       class string: { },
       class wstring: { },
       template pair,
       template tuple,
       template initializer_list,
       template allocator,
       template equal_to,
       template not_equal_to,
       template less,
       template greater,
       template less_equal,
       template greater_equal,
       template deque,
       class tm: {
         member variable tm_sec,
         member variable tm_min,
         member variable tm_hour,
         member variable tm_mday,
         member variable tm_mon,
         member variable tm_year,
         member variable tm_wday,
         member variable tm_yday,
         member variable tm_isdst
       },
       template vector,
       template list,
       template set,
       template map
     },
     namespace boost: {
       template optional
     },
     namespace mirror: { },
     type void,
     type bool,
     type char,
     type unsigned char,
     type wchar_t,
     type short int,
     type int,
     type long int,
     type unsigned short int,
     type unsigned int,
     type unsigned long int,
     type float,
     type double,
     type long double
   }
\end{verbatim}

For more examples of usage see ~\cite{mirror-doc-puddle-examples}.

\subsubsection{Rubber}

Rubber is a OOP-style run-time type erasure utility built on top
of Mirror and Puddle. It again follows the metaobject concept hierarchy of Mirror and Puddle.
Rubber allows to access and store metaobjects of the same category in a single
type, so in contrast to Mirror and Puddle where a meta-type reflecting the \verb@int@
type and a meta-type reflecting the \verb@double@ type have different types
in Rubber they can both be stored in a variable of the same type.
Rubber does not use virtual functions but rather pointers to existing
functions implemented by Mirror to achieve run-time polymorphism.

The first example shows the usage of type-erased metaobjects with a C++11
lambda function which could not be used with Mirror's or Puddle's meda-objects
(because lambdas are not templated):

\begin{lstlisting}
#include <mirror/mirror.hpp>
#include <rubber/rubber.hpp>
#include <iostream>

int main(void)
{
    // use the Mirror's for_each function, but erase the types
    // of the iterated compile-time metaobjects before passing
    // them as arguments to the lambda function.
    mirror::mp::for_each<
        mirror::members<
            MIRRORED_GLOBAL_SCOPE()
        >
    >(
        // the rubber::meta_named_scoped_object type is
        // constructible from a Mirror MetaNamedScopedObject
        [](const rubber::meta_named_scoped_object& member)
        {
            std::cout <<
                member.self().construct_name() <<
                " " <<
                member.base_name() <<
                std::endl;
        }
    );
    return 0;
}
\end{lstlisting}

This simple application prints the following on the standard output:

\begin{verbatim}
namespace std
namespace boost
namespace mirror
type void
type bool
type char
type unsigned char
type wchar_t
type short int
type int
type long int
type unsigned short int
type unsigned int
type unsigned long int
type float
type double
type long double
\end{verbatim}

The next example prints different information for different categories
of metaobjects:

\begin{lstlisting}
#include <mirror/mirror.hpp>
#include <rubber/rubber.hpp>
#include <iostream>
#include <vector>

int main(void)
{
    using namespace rubber;
    mirror::mp::for_each<
        mirror::members<
            MIRRORED_GLOBAL_SCOPE()
        >
    >(
        eraser<meta_scope, meta_type, meta_named_object>(
            [](const meta_scope& scope)
            {
                std::cout <<
                    scope.self().construct_name() <<
                    " '" <<
                    scope.base_name() <<
                    "', number of members = " <<
                    scope.members().size() <<
                    std::endl;
            },
            [](const meta_type& type)
            {
                std::cout <<
                    type.self().construct_name() <<
                    " '" <<
                    type.base_name() <<
                    "', size in bytes = " <<
                    type.sizeof_() <<
                    std::endl;
            },
            [](const meta_named_object& named)
            {
                std::cout <<
                    named.self().construct_name() <<
                    " '" <<
                    named.base_name() <<
                    "'" <<
                    std::endl;
            }
        )
    );
    return 0;
}
\end{lstlisting}

It has the following output:

\begin{verbatim}
namespace 'std', number of members = 20
namespace 'boost', number of members = 0
namespace 'mirror', number of members = 0
type 'void', size in bytes = 0
type 'bool', size in bytes = 1
type 'char', size in bytes = 1
type 'unsigned char', size in bytes = 1
type 'wchar_t', size in bytes = 4
type 'short int', size in bytes = 2
type 'int', size in bytes = 4
type 'long int', size in bytes = 8
type 'unsigned short int', size in bytes = 2
type 'unsigned int', size in bytes = 4
type 'unsigned long int', size in bytes = 8
type 'float', size in bytes = 4
type 'double', size in bytes = 8
type 'long double', size in bytes = 16
\end{verbatim}

For more examples of usage see ~\cite{mirror-doc-rubber-examples}.

\subsubsection{Lagoon}

Lagoon defines run-time polymorphic interfaces and classes implementing these
interfaces and wrapping the compile-time metaobjects from Mirror and Puddle.
While Rubber is more suitable for simple decoupling of reflection-based
algorithms from the real types of the metaobjects that the algorithms
operate on, Lagoon is full-blown run-time reflection utility that can be
even decoupled from the application using it and loaded dynamically on-demand.

See appendix~\ref{appendix-lagoon-examples} for examples of usage.

