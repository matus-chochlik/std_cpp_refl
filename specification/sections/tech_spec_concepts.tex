\subsection{Metaobject Concepts}

This section conceptualy describes the requirements that various metaobjects
need to satisfy in order to be considered models of the individual
concepts. There are several ways how the conceptual model
can be transformed into the final C++ form. These are described in the \ref{concept-rendering} subsection below.
For examples of concrete renderings please see Appendix \ref{appendix-concept-hierarchy}.

\subsubsection{Categorization and Traits}

In order to provide means for distinguishing between regular types
and metaobjects the \verb@is_metaobject@ trait should be added
and should "return" \verb@true@ for metaobjects (types defined
by the compiler providing metadata) and \verb@false@
for non-metaobjects (native or user defined types).
See the definition of the {\metaobject{Metaobject}} concept for further 
metaobject traits and tags.

\input{sections/tech_spec_concept_defs.tex}
\subsection{Reflection Functions}

The metaobject should be provided via a set of overloaded
functions defined in the \verb@std@ namespace.

\subsubsection{Global scope reflection}

\subsubsection{Type reflection}


\subsection{Tagging}

As described above tagging can provide useful
additional information about program features
that are reflected by meatobjects to reflection-based
metaprograms


