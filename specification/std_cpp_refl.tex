\documentclass[11pt,a4paper,oneside]{scrartcl}

\newcommand{\mchname}{Mat\'{u}\v{s} Chochl\'{i}k}
\newcommand{\mchmail}{Matus.Chochlik@fri.uniza.sk}
\newcommand{\docnum}{Dnnnn=14-mmmm}
\newcommand{\docdate}{2014-05-14}

\usepackage[utf8]{inputenc}
\usepackage{url}
\usepackage[colorlinks=true]{hyperref}
\usepackage{parskip}

\usepackage{listings}
\usepackage{minted}

\usepackage{fancyhdr}
\setlength{\headheight}{14pt}
\pagestyle{fancyplain}
\lhead{\fancyplain{}{ISO/IEC JTC1 SC22 WG21 \docnum{} - Static reflection}}
\rhead{}
\rfoot{\fancyplain{}{\thepage}}
\cfoot{}

\usepackage[pdftex]{graphicx}
\DeclareGraphicsExtensions{.pdf,.png,.jpg,.mps,.eps}
\graphicspath{{images/}}

\usepackage{tikz}
\usetikzlibrary{arrows,positioning}
\tikzstyle{concept}=[
	rectangle,	
	very thick,
	draw=red!80!black!80,
	top color=white,
	bottom color=red!20,
	node distance=0.5em and 1.5em
]
\tikzstyle{inheritance}=[
	->,
	shorten >=1pt,
	>=open triangle 90,
	very thick
]
   

\setcounter{tocdepth}{3} 

\title{Static reflection}

\author{\mchname (\mchmail)}

\newcommand{\metaobject}[1]{\hyperref[concept-def-#1]{\em{#1}}}

\begin{document}

\begin{tabular}{r l}
Document number: & \docnum\\
Date: & \docdate\\
Project: & Programming Language C++, Library Working Group \\
Reply-to: & \mchname (\href{mailto:\mchmail}{\mchmail})\\
\end{tabular}

\paragraph{How to read this document}
The first two sections are devoted to the introduction to reflection and reflective
programming, they contain some motivational examples and some experiences
with usage of a library-based reflection utility. These can be skipped
if you are knowledgeable about reflection.
Section \ref{section-design} contains the rationale for the design decisions. 
The most important part is the technical specification in section \ref{section-tech-spec},
the impact on the standard is discussed in section \ref{section-impact},
the issues that need to be resolved are listed in section \ref{section-issues},
and section \ref{section-impl-hints} mentions some implementation hints.

\tableofcontents

\section{Introduction}

Reflection and reflective programming can be used
for a wide range of tasks such as implementation of serialization-like operations,
remote procedure calls, scripting, automated GUI-generation,
implementation of several software design patterns, etc.
C++ as one of the most prevalent programming languages 
lacks a standardized reflection facility.

In this paper we propose the addition of native support for
compile-time reflection to C++ and a library built
on top of the metadata provided by the compiler.

The basic static metadata provided by compile-time reflection
should be as complete as possible to be applicable in a wide
range of scenarios and allow to implement custom higher-level
static and dynamic reflection libraries and reflection-based
utilities.

The term \emph{reflection} refers to the ability of a computer program
to observe and possibly alter its own structure and/or its behavior.
This includes building new or altering the existing data structures,
doing changes to algorithms or changing the way the program code
is interpreted. Reflective programming is a particular kind
of \emph{metaprogramming}.

The advantage of using reflection is in the fact that everything
is implemented in a single programming language, and the human-written
code can be closely tied with the customizable reflection-based
code which is automatically generated by compiler metaprograms,
based on the metadata provided by reflection.

The solution proposed in this paper is based on the
\href{http://kifri.fri.uniza.sk/~chochlik/mirror-lib/html/}{\em Mirror}
reflection utilities~\cite{mirror-doc-cpp11} and on several years
of user experience with reflection-based metaprogramming.


\section{Motivation and Scope}

\subsection{Usefullness of reflection}

There is a wide range of computer programming tasks that involve
the execution of the same algorithm on a set of types defined by an
application or on instances of these types, accessing member variables,
calling free or member functions in an uniform
manner, converting data between the language's intrinsic representation and
external formats for the purpose of implementing the following:

\begin{itemize}

\item serialization or storing of persistent data in a
custom binary format or in XML, JSON, XDR, etc.,

\item (re-)construction of class instances
from external data representations (like those listed above),
from the data stored in a relational database, from data entered by
a user through a user interface or queried through a web service API,

\item automatic generation of a relational schema from the application
object model and object-relational mapping (ORM),

\item support for scripting 

\item support remote procedure calls (RPC) / remote method invocation (RMI),

\item inspection and manipulation of existing objects via a (graphic) user interface
or a web service,

\item visualization of objects or data and the relations between objects or
relations in the data,

\item automatic or semi-automatic implementation of certain software design patterns,

\item etc.

\end{itemize}

There are several aproaches to the implementation of such
functionality. The most straightforward and also usually the most
error-prone is manual implementation. Many of the tasks listed above
are inherently repetitive and basically require to process programming
language constructs (types, structures, containers, functions, constructors,
class member variables, enumerated values, etc.)
in a very uniform way that could be easily transformed into a meta-algorithm.

While it is acceptable (even if not very advantageous)
for example, for a design pattern implementation to be made by a human,
writing RPC/RMI-related code is a task much better suited for a computer.

This leads to the second, heavily used approach: preprocessing
and parsing of the program source text by a (usually very specfic) external
program (documentation generation tool, interface definition language compiler
for RPC/RMI, web service interface generator, a rapid application
development environment with a form designer, etc.) resulting in additional
program source code, which is then compiled into the final application binary.

This approach has several problems, first it requires the external
tools which may not fit well into the build system or may not be portable
between platforms or be free; second, such tools are task-specific
and many of them allow only a limited, if any, customization of the output.

Another way to automate these tasks is to use reflection,
reflective programming, metaprogramming and generic programming as
explained below.


\section{Design Decisions}

\subsection{Compile-time vs. Run-time reflection}

Run-time, dynamic reflection facilities may seem more readily
usable, but with the increasing popularity of compile-time metaprogramming,
the need for compile-time introspection (already taken care of
by \verb@type_traits@) and reflection also increases.

From the performance point of view, algorithms based on static
meta-data offer much more possibilities for the compiler to do
optimizations.

Also, if compile-time reflection is well supported it is relatively
easy to implement run-time or even dynamically loadable reflection
on top of it.

Thus, taking shortcuts directly to run-time reflection, without
compile-time support has obvious drawbacks.

\subsection{Desired features} 

The proposed reflection facility is designed with the following
goals in mind:

\begin{itemize}
\item {\em Reusability}: The provided metadata should be reusable
in many situations and for many different purposes, not only
the obvious ones. This is closely related to {\em completeness} (below).

\item {\em Flexibility}: The basic reflection and the libraries
built on top of it should be designed
in a way that they are eventually usable during both compile-time
and run-time and under various paradigms (object-oriented, functional, etc.),
depending on the application needs.

\item {\em  Encapsulation}: The metadata should be accessible
through conceptually well-defined interfaces.

\item {\em  Stratification}: Reflection should be non-intrusive,
and the meta-level should be separated from the base-level language
constructs it reflects. Also, reflection should not be implemented
in a all-or-nothing manner. Things that are not needed, should not generally
be compiled-into the final application.

\item {\em  Ontological correspondence}: The meta-level facilities should
correspond to the ontology of the base-level C++ language constructs
which they reflect. This basically means that all existing language
features should be reflected and new ones should not be invented.
This rule may have some important exceptions like the reflection of
containers.

\item {\em  Completeness}: The proposed reflection facility should
provide as much useful metadata as possible, including various specifiers,
(like constness, storage-class, access, etc.), namespace members,
enumerated types, iteration of namespace members and much more.

\item {\em  Ease of use}: Although reflection-based metaprogramming
allows to implement very complicated things, simple things
should be kept simple.

\item {\em  Cooperation with other librares}: Reflection should be
usable with the existing introspection facilites (like \verb@type_traits@)
already provided by the standard library and with other libraries.
\end{itemize}


\section{Technical Specifications}

We propose that the basic metadata describing a program written
in C++ should be made available through a set of {\em anonymous} classes
defined by the compiler. These classes should describe various program
constructs like, namespaces, types, typedefs, classes, their member variables
(member data), member functions, inheritance, templates, template parameters,
enumerated values, etc.

The compiler should generate metadata for the program constructs defined
in the currently processed translation unit. Indexed sets of metaobjects,
like scope members, parameters of a function, etc. should be listed
in the order of appearance in the processed source code.

Since we want the metadata to be available at compile-time,
different base-level constructs should be reflected by
{\em "statically" different} metaobjects and thus by {\em different} types.
For example a metaobject reflecting the global scope namespace should
be a different {\em type} than a metaobject reflecting the \verb@std@
namespace, a metaobject reflecting the \verb@int@ type should
have a different type then a metaobject reflecting the \verb@double@
type, a metaobject reflecting \verb@::foo(int)@ function should
have a different type than a metaobject reflecting \verb@::foo(double)@,
function, etc.

In a manner of speaking these special types (metaobjects) should become
"instances" of the meta-level concepts (static interfaces which
should not exist as concrete types, but rather only at the
"specification-level" similar for example to the iterator concepts).
This section describes a set of metaobject concepts,
their interfaces, tag types for metaobject classification and
functions (or operators) providing access to the metaobjects.

\subsection{Metaobject Concepts}

This section describes the requirements that various metaobjects
need to satisfy in order to be considered models of the individual
concepts.

\subsubsection{Categorization and Traits}

In order to provide means for distinguishing between regular types
and metaobjects the \verb@is_metaobject@ trait should be added
and should "return" \verb@true_type@ for metaobjects (types defined
by the compiler providing metadata) and \verb@false_type@
for non-metaobjects (native or user defined types).

The \verb@metaobject_traits@ structure should be defined to provide
categorization and additional information about the interface of metaobjects.

\begin{lstlisting}
template <typename Metaobject>
struct metaobject_traits
{
	typedef typename Metaobject::category category;

	typedef Bool has_name;

	typedef Bool has_scope;

	typedef Bool has_members;

	typedef Bool has_template;
};
\end{lstlisting}

The meaning of the individual trait typedefs is following:

\begin{itemize}
\item{\verb@category@} Is one of the following types and specifies the category
of the metaobject:
	\begin{itemize}
		\item{\verb@specifier_tag@} indicates a {\metaobject Specifier}.
		\item{\verb@namespace_tag@} indicates a {\metaobject Namespace}.
		\item{\verb@type_tag@} indicates a {\metaobject Type}.
		\item{\verb@typedef_tag@} indicates a {\metaobject Typedef}.
		\item{\verb@class_tag@} indicates a {\metaobject Class}.
		\item{\verb@function_tag@} indicates a {\metaobject Function}.
		\item{\verb@constructor_tag@} indicates a {\metaobject Constructor}.
		\item{\verb@operator_tag@} indicates an {\metaobject Operator}.
		\item{\verb@overloaded_function_tag@} indicates an {\metaobject OverloadedFunction}.
		\item{\verb@template_tag@} indicates a {\metaobject Template}.
		\item{\verb@enum_tag@} indicates an {\metaobject Enum}.
		\item{\verb@enum_value_tag@} indicates an {\metaobject EnumValue}.
		\item{\verb@inheritance_tag@} indicates an {\metaobject Inheritance}.
		\item{\verb@variable_tag@} indicates a {\metaobject Variable}.
		\item{\verb@parameter_tag@} indicates a {\metaobject Parameter}.
	\end{itemize}

\item{\verb@has_name@} indicates that the reflected object is {\metaobject Named}.
\item{\verb@has_scope@} indicates that the reflected object is {\metaobject Scoped}.
\item{\verb@has_members@} indicates that the reflected object is a {\metaobject Scope}.
\item{\verb@has_template@} indicates that the reflected object is {\metaobject Templated}.
\end{itemize}

\subsubsection{Metaobject}

{\metaobject Metaobject} is a stateless anonymous \verb@struct@ that provides
metadata reflecting certain program constructs and has the following properties:

\begin{itemize}
\item For every {\metaobject Metaobject} the \verb@is_metaobject@ trait returns \verb@true_type@.
\item For every {\metaobject Metaobject} the \verb@metaobject_traits@ structure is defined.
\item For every {\metaobject Metaobject} the {\verb@typedef Metaobject::category@} is defined
and has the same meaning as \verb@metaobject_category<Metaobject>::category@.
\end{itemize}

The exact type of a specific {\metaobject Metaobject} reflecting a specific
program feature is not defined by the standard, instances of metaobjects
should be always declared through the \verb@auto@ type specifier.

All instances of a specific {\metaobject Metaobject} should be equal to
the programmer and no internal context should be visible on the outside.

\subsubsection{Specifier}

{\metaobject Specifier} is a {\metaobject Metaobject}, which reflects specifiers like
\verb@const@, \verb@volatile@, \verb@private@,
\verb@protected@, \verb@public@, \verb@virtual@, etc. and has the following
interface:

\begin{itemize}

\item{\verb@static const char* keyword(void);@} returns the keyword
of the reflected specifier. If \verb@category@ is \verb@spec_none_tag@
then \verb@keyword@ returns "" (an empty c-string).

\item{\verb@typedef Category category;@} is defined as one of the following 
types:
	\begin{itemize}
		\item{\verb@spec_none_tag@} a category for missing specifiers,
		for example a non-const member function would have a \verb@spec_none_tag@
		constness specifier or a variable with automatic storage class
		would have a \verb@spec_none_tag@ storage class specifier, etc.

		\item{\verb@spec_extern_tag@} indicates \verb@extern@ storage class / linkage.
		\item{\verb@spec_static_tag@} indicates \verb@static@ storage class / linkage.
		\item{\verb@spec_mutable_tag@} indicates \verb@mutable@ storage class / linkage.
		\item{\verb@spec_register_tag@} indicates \verb@register@ storage class / linkage.
		\item{\verb@spec_thread_local_tag@} indicates \verb@thread_local@ storage class / linkage.

		\item{\verb@spec_const_tag@} indicates \verb@const@ member functions.

		\item{\verb@spec_virtual_tag@} indicates \verb@virtual@ inheritance or function linkage.

		\item{\verb@spec_private_tag@} indicates \verb@private@ member access.
		\item{\verb@spec_protected_tag@} indicates \verb@protected@ member access.
		\item{\verb@spec_public_tag@} indicates \verb@public@ member access.

		\item{\verb@spec_class_tag@} indicates the \verb@class@ elaborated type specifier.
		\item{\verb@spec_struct_tag@} indicates the \verb@struct@ elaborated type specifier.
		\item{\verb@spec_union_tag@} indicates the \verb@union@ elaborated type specifier.
		\item{\verb@spec_enum_tag@} indicates the \verb@enum@ elaborated type specifier.
	\end{itemize}
\end{itemize}

\subsubsection{Named}

{\metaobject Named} is a {\metaobject Metaobject} reflecting program constructs,
which have a name, like namespaces, types, functions, variables, etc. {\metaobject Named}
metaobjects add the following functions to the {\metaobject Metaobject} interface:

\begin{itemize}

	\item{\verb@static const char* base_name(void);@} returns the base name
	of the reflected construct, without the nested name specifier. For namespace
	\verb@std@ this function should return "std", for namespace \verb@foo::bar::baz@
	this function should return "baz", for the global scope this function
	should return "" (an empty c-string literal).\\For \verb@std::vector<int>::iterator@
	it should return "iterator". For derived and qualified types like \\
	\verb@volatile std::vector<const foo::bar::fubar*> * const *@ it should return
	"volatile vector$<$const fubar*$>$ * const *", etc.

	\item{\verb@static const char* full_name(void);@} returns the full name
	of the reflected construct, with the nested name specifier. For namespace
	\verb@std@ this function should return "std", for namespace \verb@foo::bar::baz@
	this function should return "foo::bar::baz", for the global scope this function
	should return "" (an empty c-string literal).\\For \verb@std::vector<int>::iterator@
	it should return "std::vector$<$int$>$::iterator". For derived and qualified types like\\
	\verb@volatile std::vector<const foo::bar::fubar*> * const *@ it should return
	"volatile std::vector$<$const foo::bar::fubar*$>$ * const *", etc. For some
	metaobjects this function may return the same value as the \verb@base_name@ function.
\end{itemize}

The following is also true for {\metaobject Named} metaobjects.

\begin{itemize}
	\item \verb@metaobject_traits<Named>::has_name@ is defined as \verb@true_type@.
\end{itemize}

\subsubsection{Scoped}

{\metaobject Scoped} is a {\metaobject Metaobject} reflecting program constructs,
which are defined inside a scope (global scope, namespace, class, etc.). {\metaobject Scoped}
metaobjects have the following interface:

\begin{itemize}
	\item{\verb@typedef Scope scope;@} defined as a {\metaobject Scope} metaobject
	reflecting the scope of the scoped object.
\end{itemize}

The following is also true for {\metaobject Scoped} metaobjects.

\begin{itemize}
	\item \verb@metaobject_traits<Scoped>::has_scope@ is defined as \verb@true_type@.
\end{itemize}

\subsubsection{Scope}

{\metaobject Scope} is a {\metaobject Named} and {\metaobject Scoped} metaobject,
which reflects scopes like namespaces, classes, enums, etc. {\metaobject Scope}
has the following interface:

\begin{itemize}

	\item{\verb@typedef integral_constant<int,@ {\em number-of-scope-members}
	\verb@>@\\\verb@member_count;@} the total number of various members like types,
	namespace, functions, variables, etc. defined inside
	the scope reflected by a {\em Scope}.

	\item{\verb@static @{\em Scoped}\verb@ member(integral_constant<int, @{\em i}
	\verb@>);@} defined for $i \in \{0, 1, \dots, n-1\}$; {\em n = number-of-scope-members},
	each overload returns a different {\metaobject Scoped} metaobject reflecting the {\em i}-th member
	defined inside the scope reflected by a {\metaobject Scope}.
\end{itemize}

The following is also true for {\metaobject Scope} metaobjects.

\begin{itemize}
	\item \verb@metaobject_traits<Scope>::has_members@ is defined as \verb@true_type@.
\end{itemize}

\subsubsection{Namespace}

{\metaobject Namespace} is a {\metaobject Scope} for which the following is true:

\begin{itemize}
	\item \verb@metaobject_traits<Namespace>::category@ is defined as
	\verb@namespace_tag@.
\end{itemize}

\subsubsection{Type}

{\metaobject Type} is a {\metaobject Named} and {\metaobject Scoped} metaobject which
has the following interface:

\begin{itemize}
	\item{\verb@typedef @{\em original-type}\verb@ original_type;@} defined as the original type
	reflected by the {\metaobject Type}.
\end{itemize}

The following is also true for {\metaobject Type} metaobjects.

\begin{itemize}
	\item \verb@metaobject_traits<Type>::category@ is defined as \verb@type_tag@.
\end{itemize}

\subsubsection{Typedef}

{\metaobject Typedef} is a {\metaobject Type} metaobject that reflects typedefs,
i.e. types that were defined as alternate names for another types.
{\metaobject Typedef} adds the following to the interface of {\metaobject Type}:

\begin{itemize}
	\item{\verb@typedef @{\metaobject Type}\verb@ type;@} defined as the {\metaobject Type}
	reflecting the "source" type of the typedef.
\end{itemize}

The following is also true for {\metaobject Typedef} metaobjects.

\begin{itemize}
	\item \verb@metaobject_traits<Typedef>::category@ is defined as \verb@typedef_tag@.
\end{itemize}

\subsubsection{Class}

{\metaobject Class} is a {\metaobject Type} and a {\metaobject Scope} that reflects
an elaborated type (class, struct, union). {\metaobject Class} has the following interface:

\begin{itemize}
	\item{\verb@typedef @{\metaobject Specifier}\verb@ elaborated_type;@} defined as
	a {\metaobject Specifier} reflecting the elaborated type specifier used
	to define the class (\verb@class@, \verb@struct@, \verb@union@).
\end{itemize}

The following is also true for {\metaobject Class}.

\begin{itemize}
	\item \verb@metaobject_traits<Class>::category@ is defined as \verb@class_tag@.
\end{itemize}

\subsubsection{Function}

{\metaobject Function} is a {\metaobject Scope} metaobject that reflects a function.

\begin{itemize}
	\item{\verb@typedef @{\metaobject Type}\verb@ result_type;@} defined as
	a {\metaobject Type} reflecting the result type of the function.

	\item{\verb@typedef @{\metaobject Specifier}\verb@ linkage;@} defined as 
	a {\metaobject Specifier} reflecting the linkage specifier of the function.
\end{itemize}

The following is also true for {\metaobject Function} metaobjects.

\begin{itemize}
	\item \verb@metaobject_traits<Function>::category@ is defined as \verb@function_tag@.
\end{itemize}

\subsubsection{ClassMember}

{\metaobject ClassMember} is a {\metaobject Named} and {\metaobject Scoped} metaobject
that is a member of a class. It has the following interface:

\begin{itemize}
	\item{\verb@typedef @{\metaobject Specifier}\verb@ access_type;@} defined as
	a {\metaobject Specifier} reflecting the access type specifier of ther class member
	(\verb@private@, \verb@protected@ or \verb@public@).
\end{itemize}

The following is also true for {\metaobject ClassMember} metaobjects.

\begin{itemize}
	\item \verb@metaobject_traits<ClassMember::scope>::category@ is \verb@class_tag@.
\end{itemize}

\subsubsection{Constructor}

{\metaobject Constructor} is a {\metaobject ClassMember} and {\metaobject Function} that
reflects a constructor.

\subsubsection{Operator}

\subsubsection{OverloadedFunction}

\subsubsection{Template}

\subsubsection{Templated}

\subsubsection{Enum}

\subsubsection{EnumValue}

\subsubsection{Inheritance}

\subsubsection{Variable}

\subsubsection{Parameter}

\section{Impact On the Standard}

\section{Implementation hints}

\subsection{Generation of metaobjects}

The metaobjects should be generated / instantiated by the compiler only
when explicitly requested. This also applies to members of the metaobjects. For example when a {\metaobject Namespace}
reflecting the \verb@std@ namespace is generated the individual \verb@member(...)@
functions (and the resulting metaobjects) should {\em not} be generated automatically
unless the \verb@Scope::member(...)@ function is called or its type queried (by \verb@decltype@
or otherwise).

This should probably improve the compilation times and avoid reflection-related
overhead when reflection is not used.

\section{Unresolved Issues}

\begin{itemize}
	\item {\em Normalization of names returned by \verb@Named::base_name()@\\and \verb@Named::full_name()@:}
	The strings returned by the \verb@base_name@ and \verb@full_name@ functions should be
	implementation-independent and the same on every platform/compiler.
	
	\item {\em Returning names as compile-time strings:} It would be advantageous if even
	the names of various metaobjects were compile-time constants and could be introspected
	or used as template parameters. See for example the Mirror's compile-time strings~\cite{mirror-ct-strings}.

	\item {\em Annotation base-level program constructs with tags and relations}.

	\item {\em Explicit specification of what should eb reflected}. It might be useful to have
	the ability to explicitly specify either what to reflect or what to hide from reflection.
	This might be a separate feature, but it also could be merged with the tagging functionality.
\end{itemize}

\input{sections/acknowl.tex}

\renewcommand\refname{\arabic{section}\hspace{1em}References}

\stepcounter{section}
\addcontentsline{toc}{section}{\refname}

\begin{thebibliography}{100}

\bibitem{mirror-doc-cpp11}
Mirror C++ reflection utilities (C++11 version),
\url{http://kifri.fri.uniza.sk/~chochlik/mirror-lib/html/}.

\bibitem{mirror-doc-mirror-examples}
Mirror - Examples of usage,
\url{http://kifri.fri.uniza.sk/~chochlik/mirror-lib/html/doxygen/mirror/html/examples.html}.

\bibitem{mirror-doc-puddle-examples}
Mirror - The Puddle layer - examples of usage,
\url{http://kifri.fri.uniza.sk/~chochlik/mirror-lib/html/doxygen/puddle/html/examples.html}.

\bibitem{mirror-doc-rubber-examples}
Mirror - The Rubber layer - examples of usage,
\url{http://kifri.fri.uniza.sk/~chochlik/mirror-lib/html/doxygen/rubber/html/examples.html}.

\bibitem{mirror-doc-lagoon-examples}
Mirror - The Lagoon layer - examples of usage,
\url{http://kifri.fri.uniza.sk/~chochlik/mirror-lib/html/doxygen/lagoon/html/examples.html}.

\bibitem{mirror-ct-strings}
Mirror - Compile-time strings
\url{http://kifri.fri.uniza.sk/~chochlik/mirror-lib/html/doxygen/mirror/html/d5/d1d/group__ct__string.html}.


\end{thebibliography}



\appendix
\section{Concept hierarchy}
\label{appendix-concept-hierarchy}

\subsection{String}

\begin{minted}{cpp}
class String
{
public:
	static constexpr const char* c_str(void);

	static constexpr size_t size(void);	
};

static_assert(
	String::c_str() != nullptr,
	"Undefined string"
);
\end{minted}

\subsection{MetaobjectCategory}

\begin{minted}{cpp}
struct metaobject_tag
{ };

struct specifier_tag
 : metaobject_tag
{ };

struct spec_extern_tag
 : specifier_tag
{ };

struct spec_static_tag
 : specifier_tag
{ };

struct spec_mutable_tag
 : specifier_tag
{ };

struct spec_register_tag
 : specifier_tag
{ };

struct spec_thread_local_tag
 : specifier_tag
{ };

struct spec_thread_local_tag
 : specifier_tag
{ };

struct spec_const_tag
 : specifier_tag
{ };

struct spec_virtual_tag
 : specifier_tag
{ };

struct spec_private_tag
 : specifier_tag
{ };

struct spec_protected_tag
 : specifier_tag
{ };

struct spec_public_tag
 : specifier_tag
{ };

struct spec_class_tag
 : specifier_tag
{ };

struct spec_struct_tag
 : specifier_tag
{ };

struct spec_union_tag
 : specifier_tag
{ };

struct spec_enum_tag
 : specifier_tag
{ };

struct namespace_tag
{ };

struct global_scope_tag
{ };

struct type_tag
{ };

struct typedef_tag
{ };

struct class_tag
{ };

struct function_tag
{ };

struct constructor_tag
{ };

struct operator_tag
{ };

struct overloaded_function_tag
{ };

struct enum_tag
{ };

struct inherintance_tag
{ };

struct constant_tag
{ };

struct variable_tag
{ };

struct parameter_tag
{ };

// all the types listed above are models
// of the MetaobjectCategory concept
typedef unspecified MetaobjectCategory;
\end{minted}

\subsection{Metaobject}

\begin{minted}{cpp}
struct Metaobject
{
	typedef MetaobjectCategory category;
};

static_assert(
	is_metaobject<Metaobject>::value,
	"Not a valid metaobject"
);

static_assert(
	is_same<
		Metaobject::category,
		metaobject_traits<Metaobject>::category
	>::value,
	"Category mismatch"
);
\end{minted}

\subsection{Specifier}

\begin{minted}{cpp}
struct Specifier
 : Metaobject
{
	typedef specifier_tag category;

	static constexpr String keyword(void);
};
\end{minted}

\subsection{Named}

\begin{minted}{cpp}
struct Named
 : Metaobject
{
	typedef specifier_tag category;

	static constexpr String keyword(void);
};
\end{minted}



\begin{minted}{cpp}
\end{minted}

\section{Implementation of String}

This appendix shows how the compile-time string concept could be implemented.

\begin{minted}{cpp}

\end{minted}

Example of usage:

\begin{minted}{cpp}
\end{minted}


TODO: to be finished.


\end{document}
