\subsection{Specifier reflection}
\label{fut-specifiers}

There are several use-cases where the reflection of specifiers would
be useful.

We might want for example to \say{copy} the storage class specifier of
one variable to other variables

\begin{minted}[tabsize=4]{cpp}
thread_local int dummy_v;
using m_storage = get_storage_class_specifier_t<reflexpr(dummy_v)>;

unreflexpr(m_storage) long foo_v;   // thread_local long foo_v;
unreflexpr(m_storage) double bar_v; // thread_local double bar_v;
unreflexpr(m_storage) short baz_v;  // thread_local short baz_v;
\end{minted}

or the linkage specifier from one function
to several other functions:

\begin{minted}[tabsize=4]{cpp}
static inline void dummy_f(void);
using m_linkage = get_linkage_specifier_t<reflexpr(dummy_f)>;

unreflexpr(m_linkage) volatile int foo_f(int);  // static volatile int foo_f(int);
unreflexpr(m_linkage) long bar_f(short, short); // static long bar_f(short, short);
\end{minted}

The open question at the moment is whether we should allow \say{direct}
reflection of specifiers since some keywords like \verb@inline@ or \verb@virtual@
have several different meanings:

\begin{minted}[tabsize=4]{cpp}
#ifdef IMPLEMENT
using m_linkage = reflexpr(static);
#else
using m_linkage = reflexpr(extern);
#endif

unreflexpr(m_linkage) volatile int foo_f(int);
unreflexpr(m_linkage) long bar_f(short, short);
\end{minted}

