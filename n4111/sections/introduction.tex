\section{Introduction}

Reflection and reflective programming can be used
for a wide range of tasks, such as the implementation
of serialization-like operations, object-relational mapping,
remote procedure calls, scripting, automated GUI-generation,
implementation of several software design patterns, etc.
C++ as one of the most prevalent programming languages 
lacks a standardized reflection facility.

In this paper we propose to add native support for
compile-time reflection to C++ by the means of compiler generated
types providing basic metadata describing various program constructs.
These metaobjects, together with some additions to the standard
library can later be used to implement other third-party libraries
providing both compile-time and run-time high-level
reflection utilities.

The basic static metadata provided by compile-time reflection
should be as complete as possible to be applicable in a wide
range of scenarios and allow to implement custom higher-level
static and dynamic reflection libraries and reflection-based
utilities.

\subsection{Differences between n3996 and n4111}

This proposal (n4111) is a conceptual successor of n3996 \cite{n3996}, but it has
been rewritten nearly from scratch (some parts of the text were reused).
The main differences are listed here:

\begin{itemize}
\item n3996 included a brief introduction to reflection, a discussion
of its usefulness and presented several use-cases.
\item n3996 included multiple examples from the Mirror reflection utilities
\cite{mirror-doc-cpp11}, for example the factory generator.
\item In n3996, the metaobject concepts were described just conceptually
and a possible renderings of the concepts into valid C++ were discussed.
In n4111 the definition of the concepts is more detailed, specific and includes
concrete definitions of what should be added to the standard and hints
how certain concepts could be implemented.
\item The definition of several concepts was changed. For example
  \begin{itemize}
  \item the requirements for the \meta{NamedScoped} concept were moved
  to the \meta{Named} concept,
  \item the string which is the result of the \verb@base_name@ template
  class has been changed,
  \item the definition of the \verb@named_typedef@ and the \verb@named_mem_var@
  template classed is more detailed,
  \item some of the metaobject attributes (rendered as template classes)
  were renamed.
  \end{itemize}
\item This proposal includes additions to the standard library.
\item The appendices in n4111 include concrete examples of usage
of the proposed metaobjects and the added template classes.
\end{itemize}
