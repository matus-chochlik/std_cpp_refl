\subsection{MetaScope}
\label{concept-MetaScope}

\begin{tikzpicture}
\node [concept] (Metaobject) {Metaobject};
\node [concept] (MetaNamed)[above right=of Metaobject] {MetaNamed}
	edge [inheritance, bend right] (Metaobject);
\node [concept] (MetaScoped)[below right=of Metaobject] {MetaScoped}
	edge [inheritance, bend left] (Metaobject);
\node [concept] (MetaNamedScoped)[below right=of MetaNamed, above right=of MetaScoped] {MetaNamedScoped}
	edge [inheritance, bend right] (MetaNamed)
	edge [inheritance, bend left] (MetaScoped);
\node [concept] (MetaScope)[right=of MetaNamedScoped] {MetaScope}
	edge [inheritance] (MetaNamedScoped);
\end{tikzpicture}

\meta{Scoped} is a \meta{NamedScoped} reflecting program constructs defined inside
of a named scope (like the global scope, a namespace, a class, etc.)

In addition to the requirements inherited from \meta{NamedScoped}, the following is required:

The \verb@is_scope@ template class specialization for a \meta{Scope} should
inherit from \verb@true_type@:

\begin{lstlisting}
template <>
struct is_scope<MetaScope>
 : true_type
{ };
\end{lstlisting}

\subsubsection{\texttt{members}}

A template class \verb@members@ should be defined and should inherit from a
\concept{MetaobjectSequence} containing \meta{NamedScoped} metaobjects reflecting
all members of the base-level scope reflected by this \meta{Scope}.

\begin{lstlisting}
template <typename T>
struct members;

template <>
struct members<MetaScope>
 : MetaobjectSequence<MetaNamedScoped>
{ };
\end{lstlisting}

