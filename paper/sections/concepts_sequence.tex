\subsection{Metaobject Sequence}
\label{concept-MetaobjectSequence}

As the name implies \concept{MetaobjectSequence}s are used to store seqences
of metaobjects.
A model of \concept{MetaobjectSequence} is a class conforming to the following:

It is a nullary metafunction returning itself:

\begin{lstlisting}
template <typename Metaobject>
struct MetaobjectSequence
{
	typedef MetaobjectSequence type;
};
\end{lstlisting}

Note that the above is only a psedo-code and
the template parameter \verb@Metaobject@ indicates here the minimal
metaobject concept which all elements in the sequence must satisfy.
For example a \verb@MetaobjectSequence<MetaConstructor>@ denotes a sequence
of metaobjects that all satisfy the \meta{Constructor} concept, etc.

\subsubsection{\texttt{size}}

A template class \verb@size@ is defined as follows:

\begin{lstlisting}
template <typename T>
struct size;

template <>
struct size<MetaobjectSequence<Metaobject>>
 : integral_type<size_t, N>
{ };
\end{lstlisting}

Where \verb@N@ is the number of elements in the sequence.

\subsubsection{\texttt{at}}

A template class \verb@at@, providing random access to metaobjects in a sequence
is defined (for values of \verb@I@ between \verb@0@ and \verb@N-1@ where \verb@N@
is the number of elements returned by \verb@size@) as follows:

\begin{lstlisting}
template <typename T, size_t I>
struct at;

template <size_t I>
struct at<MetaobjectSequence<Metaobject>, I>
 : Metaobject
{ };
\end{lstlisting}

