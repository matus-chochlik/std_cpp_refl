\documentclass[11pt,a4paper,oneside]{scrartcl}

\newcommand{\mchname}{Mat\'{u}\v{s} Chochl\'{i}k}
\newcommand{\mchmail}{chochlik@gmail.com}
\newcommand{\docname}{A case for strong static reflection}
\newcommand{\docnum}{Nxxxx}
\newcommand{\docdate}{2015-01-22}

\usepackage[utf8]{inputenc}
\usepackage{url}
\usepackage[colorlinks=true]{hyperref}
\usepackage{parskip}
\usepackage[titletoc]{appendix}

\usepackage{listings}
\usepackage{minted}
\lstset{basicstyle=\footnotesize\ttfamily,breaklines=true}

\usepackage{fancyhdr}
\setlength{\headheight}{14pt}
\pagestyle{fancyplain}
\lhead{\fancyplain{}{\docnum - \docname}}
\rhead{}
\rfoot{\fancyplain{}{\thepage}}
\cfoot{}

\usepackage[pdftex]{graphicx}
\DeclareGraphicsExtensions{.pdf,.png,.jpg,.mps,.eps}
\graphicspath{{images/}}

\usepackage{tikz}
\usetikzlibrary{arrows,positioning}
\tikzstyle{concept}=[
	rectangle,	
	very thick,
	draw=red!80!black!80,
	top color=white,
	bottom color=red!20,
	node distance=0.5em and 1.5em
]
\tikzstyle{inheritance}=[
	->,
	shorten >=1pt,
	>=open triangle 90,
	very thick
]
   

\setcounter{tocdepth}{3} 

\title{\docname}

\author{\mchname (\mchmail)}

\newcommand{\meta}[1]{{\em \textbf{Meta#1}}}

\begin{document}

\begin{tabular}{r l}
Document number: & \docnum\\
Date: & \docdate\\
Project: & Programming Language C++, SG7, Reflection\\
Reply-to: & \mchname (\href{mailto:\mchmail}{\mchmail})\\
\end{tabular}

\begin{center}
\vskip 2em
{\Huge \docname}
\vskip 1em
{\emph \mchname}
\vskip 2em
\end{center}

\paragraph{Abstract}

In N3996 \cite{n3996} and N4111 \cite{n4111} we proposed the design
and some hints on possible implementation of a compile-time reflection
facility for standard C++. N3996 contained list of possible use-cases
and a discussion about the usefulness of reflection. During the presentation
of the paper concerns were expressed about the level-of-detail and scope
of the presented proposal and possible dangers of giving non-expert language users
a {\em too powerful} tool to use and dangers for the implementers of this proposal.
Examples and more detailed description of use cases were called for.
This paper aims to address these issues.


\tableofcontents

\section{Introduction}

Reflection and reflective programming can be used
for a wide range of tasks such as implementation of serialization-like operations,
remote procedure calls, scripting, automated GUI-generation,
implementation of several software design patterns, etc.
C++ as one of the most prevalent programming languages 
lacks a standardized reflection facility.

In this paper we propose the addition of native support for
compile-time reflection to C++ and a library built
on top of the metadata provided by the compiler.

The basic static metadata provided by compile-time reflection
should be as complete as possible to be applicable in a wide
range of scenarios and allow to implement custom higher-level
static and dynamic reflection libraries and reflection-based
utilities.

The term \emph{reflection} refers to the ability of a computer program
to observe and possibly alter its own structure and/or its behavior.
This includes building new or altering the existing data structures,
doing changes to algorithms or changing the way the program code
is interpreted. Reflective programming is a particular kind
of \emph{metaprogramming}.

The advantage of using reflection is in the fact that everything
is implemented in a single programming language, and the human-written
code can be closely tied with the customizable reflection-based
code which is automatically generated by compiler metaprograms,
based on the metadata provided by reflection.

The solution proposed in this paper is based on the
\href{http://kifri.fri.uniza.sk/~chochlik/mirror-lib/html/}{\em Mirror}
reflection utilities~\cite{mirror-doc-cpp11} and on several years
of user experience with reflection-based metaprogramming.

\section{Basic use cases}

This sections describes trivial use-cases for reflection.

\subsection{Portable type names}

One of the notorious problems of \verb@std::type_info@ is that the string
returned by its \verb@name@ member function is not standardized and is
not even guaranteed to return any meaningful, unique human-readable string,
at least not without demangling, which is platform specific.
Furthermore the returned string is not \verb@constexpr@ and cannot be
reasoned about at compile-time and is applicable only to types.

The ability to uniquely map any type used in a program to a human-readable,
portable, compile-time string has several use-cases described below.

The \meta{Named} concept from N4111 reflects named language constructs
and provides the \verb@base_name@ and \verb@full_name@ metafunctions,
returning their basic name without any qualifiers or decorations and a fully-qualified 
portable type name.

\subsubsection{Logging}

When logging the execution of functions (especialy templated ones) it is sometimes
desirable to also include the names of the parameter types or even the names of the parameters
and other variables.

The best we can do with just the \verb@std::type_info@ is the following:

\begin{minted}{cpp}
#if __PLATFORM_ABC__
std::string demangled_type_name(const char*) { /* implementation 1 */ }
#else if __PLATFORM_MNO__
std::string demangled_type_name(const char*) { /* implementation 2 */ }
#else if __PLATFORM_XYZ__
std::string demangled_type_name(const char*) { /* implementation N */ }
#else
std::string demangled_type_name(const char* mangled_name)
{
	// don't know how to demangle this; let's try our luck
	return mangled_name;
}
#endif

template <typename T>
T min(const T& a, const T& b)
{
	log()   << "min<"
	        << demangled_type_name(typeid(T).name())
	        << ">(" << a << ", " << b << ") = ";

	T result = a<b?a:b;

	log()   << result << std::endl;

	return result;
}

\end{minted}

Which may or may not work, depending on the platform.

With the help of reflection as proposed in N4111 we could do:

\begin{minted}{cpp}
template <typename T>
T min(const T& a, const T& b)
{
	log()   << "min<"
	        << full_name<reflected(T)>()
	        << ">(" << a << ", " << b << ") = ";

	T result = a<b?a:b;

	log()   << result << std::endl;

	return result;
}
\end{minted}

\subsection{Enumeration of base classes}

\subsection{Enumeration of scope members}

\subsection{Turning compile-time strings into identifiers}


\section{Advanced use cases}


\subsection{Instantiation of objects}

Or (re-)construction of class instances
from external data representations (like those listed above),
from the data stored in a relational database, from data entered by
a user through a user interface or queried through a web service API.
TODO

\subsection{Manipulation of existing objects}
inspection and manipulation of existing objects via a (graphic) user interface
or a web service,

\subsection{Visualization}

visualization of objects or data and the relations between objects or
relations in the data,

%\subsection{Object-relational mapping}

%automatic generation of a relational schema from the application
%object model and object-relational mapping (ORM),
%TODO

%\subsection{Scripting}

%TODO

%\subsection{Remote procedure calls}

%TODO

%\subsection{Design patterns}

%Factory, ...
%TODO


\section{Design preferences}


\subsection{Consistency}

The reflection facility as a whole
should be consistent, instead of being composed of ad-hoc, individually
designed parts.

\subsection{Reusability}

The provided metadata should be reusable
in many situations and for many different purposes, not only
for the obvious use cases. This is closely related to {\em completeness} (below).

\subsection{Flexibility}

The basic reflection and the libraries
built on top of it should be designed
in a way that they are eventually usable during both compile-time
and run-time and under various paradigms (object-oriented, functional, etc.),
depending on the application needs.

\subsection{Encapsulation}

The metadata should be accessible through conceptually well-defined interfaces.
The metadata should not be exposed directly to the used by compiler built-ins, etc.

\subsection{Stratification}

Reflection should be non-intrusive,
and the meta-level should be separated from the base-level language
constructs it reflects. Also, reflection should not be implemented
in a all-or-nothing manner. Things that are not needed, should not generally
be compiled-into the final application.

\subsection{Ontological correspondence}

The meta-level facilities should
correspond to the ontology of the base-level C++ language constructs
which they reflect. This basically means that all existing language
features should be reflected and new ones should not be invented.
This rule may have some important exceptions, like the reflection of
containers. 

\subsection{Completeness}

The proposed reflection facility should
provide as much useful metadata as possible, including various specifiers,
(like constness, storage-class, access, etc.), namespace members,
enumerated types, iteration of namespace members and much more.

\subsection{Ease of use}

Although reflection-based metaprogramming
allows to implement very complicated things, simple things
should be kept simple.

\subsection{Cooperation with rest of the standard and other librares}

Reflection should be
usable with the existing introspection facilites (like \verb@type_traits@)
already provided by the standard library and with other libraries.


\renewcommand\refname{\arabic{section}\hspace{1em}References}

\stepcounter{section}
\addcontentsline{toc}{section}{\refname}

\begin{thebibliography}{100}

\bibitem{mirror-doc-cpp11}
Mirror C++ reflection utilities (C++11 version),
\url{http://kifri.fri.uniza.sk/~chochlik/mirror-lib/html/}.

\bibitem{mirror-doc-mirror-examples}
Mirror - Examples of usage,
\url{http://kifri.fri.uniza.sk/~chochlik/mirror-lib/html/doxygen/mirror/html/examples.html}.

\bibitem{mirror-doc-puddle-examples}
Mirror - The Puddle layer - examples of usage,
\url{http://kifri.fri.uniza.sk/~chochlik/mirror-lib/html/doxygen/puddle/html/examples.html}.

\bibitem{mirror-doc-rubber-examples}
Mirror - The Rubber layer - examples of usage,
\url{http://kifri.fri.uniza.sk/~chochlik/mirror-lib/html/doxygen/rubber/html/examples.html}.

\bibitem{mirror-doc-lagoon-examples}
Mirror - The Lagoon layer - examples of usage,
\url{http://kifri.fri.uniza.sk/~chochlik/mirror-lib/html/doxygen/lagoon/html/examples.html}.

\bibitem{mirror-ct-strings}
Mirror - Compile-time strings
\url{http://kifri.fri.uniza.sk/~chochlik/mirror-lib/html/doxygen/mirror/html/d5/d1d/group__ct__string.html}.


\end{thebibliography}



\end{document}
