\section{Introduction}

There is a wide range of computer programming scenarios in which
the programmer has to (or had to) manualy implement some generic boilerplate
code doing the same thing on multiple different types (or other language constructs,
like namespaces, constructors, etc.). For example accessing member variables or
calling free or member functions and operators in an uniform
manner, converting data between the language's intrinsic representation and
external formats, delegation or some of the other
established design patterns, etc.

With the right tools most of these tasks could be properly algorithmized,
encapsulated into some form of parametric {\em 'callable'} and then invoked
with appropriate arguments.

The C/C++ preprocessor and C++'s templates were devised to help dealing
with some of such cases and later gave rise to particular forms of metaprogramming,
which have since become relatively popular. However, while preprocessor and template-based
metaprogramming is being put to good use by more and more libraries and applications written
in C++, we are starting to hit the limits of these tools.

The limits stem from the fact, that while we can now use the C++'s type system
as a (meta)programming language of its own and a C++ compiler as its interpreter
and write complicated metaprograms automatically generating a lot of code
which we previously had to write manually, the programmer still needlessly has to do
a lot of things manually. Just to mention a few things; if we want portable type names,
we usually have to hardcode them, if we want to create an identifier programmatically
we need preprocessor token pasting, we cannot programmatically enumerate all member
variables or base classes of a class, all operators working on a particular type, etc.

A C++ compiler processing a translation unit has access to a large amount
of very useful metadata, but unfortunately shares only a tiny amount of it
with the programmer. Not so long ago the {\em type-traits} -- introspection primitives,
many of which rely on some sort of compiler 'magic' (a.k.a. reflection) were introduced.
Before that the only 'reflection-facility' in C++ was the \verb@typeid@ operator.

The aim of this paper is to show that we can and should be able to get much more
information from the compiler and why that would be useful. Since this paper is a follow-up
to N3996 and N4111, it also indicates how the metaobjects proposed in N4111 would
be usable in the individual use-cases.

