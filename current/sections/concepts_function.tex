\subsection{MetaFunction}
\label{concept-MetaFunction}

\begin{tikzpicture}
\node [concept] (Metaobject) {Metaobject};
\node [concept] (MetaNamed)[above right=of Metaobject] {MetaNamed}
	edge [inheritance, bend right] (Metaobject);
\node [concept] (MetaScoped)[below right=of Metaobject] {MetaScoped}
	edge [inheritance, bend left] (Metaobject);
\node [concept] (MetaNamedScoped)[below right=of MetaNamed, above right=of MetaScoped] {MetaNamedScoped}
	edge [inheritance, bend right] (MetaNamed)
	edge [inheritance, bend left] (MetaScoped);
\node [concept] (MetaFunction)[right=of MetaNamedScoped] {MetaFunction}
	edge [inheritance] (MetaNamedScoped);
\end{tikzpicture}


\meta{Function} is a \meta{NamedScoped} which reflects a function or a function template.

In addition to the requirements inherited from \meta{NamedScoped},
models of \meta{Function} are subject to the following:

The \verb@metaobject_category@ template class specialization for a \meta{Function} should
inherit from \verb@function_tag@:

\begin{minted}{cpp}
template <>
struct metaobject_category<MetaFunction>
 : function_tag
{ };
\end{minted}

If a \meta{Function} reflects a function template, then the \verb@is_template@
trait should inherit from \verb@true_type@

\subsubsection{\texttt{linkage\_specifier}}

A template class \verb@linkage_specifier@ should be defined and should inherit from
a \meta{Specifier} reflecting the linkage specifier of the function reflected by
the \meta{Function}:

\begin{minted}{cpp}
template <typename T>
struct linkage_specifier;

template <>
struct linkage_specifier<MetaFunction>
 : MetaSpecifier
{ };
\end{minted}

\subsubsection{\texttt{constexpr\_specifier}}

A template class \verb@constexpr_specifier@ should be defined and should inherit from
a \meta{Specifier} reflecting the constexpr specifier of the reflected function:

\begin{minted}{cpp}
template <typename T>
struct constexpr_specifier;

template <>
struct constexpr_specifier<MetaFunction>
 : MetaSpecifier
{ };
\end{minted}

In case the reflected function does not have the \verb@constexpr@ specifier,
then the result should be a \meta{Specifier} reflecting the "none" specifier.

\subsubsection{\texttt{result\_type}}

A template class \verb@result_type@ should be defined and should inherit from
a \meta{Type} reflecting the return value type of the reflected function:

\begin{minted}{cpp}
template <typename T>
struct result_type;

template <>
struct result_type<MetaFunction>
 : MetaType
{ };
\end{minted}

\subsubsection{\texttt{parameters}}

A template class \verb@parameters@ should be defined and should inherit from
a \concept{MetaobjectSequence} or \meta{Parameter}s reflecting the individual parameters
of the function:

\begin{minted}{cpp}
template <typename T>
struct parameters;

template <>
struct parameters<MetaFunction>
 : MetaobjectSequence<MetaParameter>
{ };
\end{minted}

\subsubsection{\texttt{noexcept\_specifier}}

A template class \verb@noexcept_specifier@ should be defined and should inherit from
a \meta{Specifier} reflecting the noexcept specifier of the reflected function:

\begin{minted}{cpp}
template <typename T>
struct noexcept_specifier;

template <>
struct noexcept_specifier<MetaFunction>
 : MetaSpecifier
{ };
\end{minted}

In case the reflected function does not have the \verb@noexcept@ specifier,
then the result should be a \meta{Specifier} reflecting the "none" specifier.

\subsubsection{\texttt{exceptions}}

A template class \verb@exceptions@ should be defined and should inherit from
a \concept{MetaobjectSequence} of \meta{Type}s reflecting the individual exception types
that the reflected function is allowed to throw:

\begin{minted}{cpp}
template <typename T>
struct exceptions;

template <>
struct exceptions<MetaFunction>
 : MetaobjectSequence<MetaType>
{ };
\end{minted}

\subsubsection{\texttt{const\_specifier}}

In case a \meta{Function} is also a \meta{ClassMember},
a template class \verb@const_specifier@ should be defined and should inherit from
a \meta{Specifier} reflecting the const specifier of the reflected member function:

\begin{minted}{cpp}
template <typename T>
struct const_specifier;

template <>
struct const_specifier<MetaFunction>
 : MetaSpecifier
{
	static_assert(is_class_member<MetaFunction>::value, "");
};
\end{minted}

In case the reflected member function does not have the \verb@const@ specifier,
then the result should be a \meta{Specifier} reflecting the "none" specifier.

\subsubsection{\texttt{is\_pure}}

In case a \meta{Function} is also a \meta{ClassMember},
a template class \verb@is_pure@ should be defined and should inherit from
\verb@true_type@ if the reflected function is pure virtual or from \verb@false_type@
otherwise:

\begin{minted}{cpp}
template <typename T>
struct is_pure;

struct is_pure<MetaFunction>
 : std::integral_constant<bool, B>
{
	static_assert(is_class_member<MetaFunction>::value, "");
};
\end{minted}

\subsubsection{\texttt{pointer}}

If the \meta{Function} reflects a namespace-level function, then
a template class \verb@pointer@ should be defined as follows:

\begin{minted}{cpp}
template <typename T>
struct pointer;

template <>
struct pointer<MetaFunction>
{
	typedef _rv_t (*type)(_param_t...);

	static type get(void);
};
\end{minted}

The \verb@get@ static member function should return a pointer to the function
reflected by the \meta{Function}.

If the \meta{Function} is also a \meta{ClassMember}, then the definition of the
\verb@pointer@ template class should be following:

\begin{minted}{cpp}
template <>
struct pointer<MetaFunction>
{
	typedef _rv_t (_cls_t::*type)(_param_t...);

	static type get(void);
};
\end{minted}

