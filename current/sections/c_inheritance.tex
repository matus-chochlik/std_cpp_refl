\subsection{MetaInheritance}
\label{concept-MetaInheritance}

\meta{Inheritance} is a \meta{Scoped} and \meta{Positional} reflecting class inheritance.

In addition to the requirements inherited from \meta{Scoped} and \meta{Positional},
types conforming to this concept must satisfy the following:

The \verb@category@ template should return \verb@inheritance_tag@ for models
of \meta{Inheritance}:

\begin{minted}{cpp}
template <>
struct category<MetaInheritance>
 : inheritance_tag
{ };
\end{minted}

The \verb@scope@ member function should inherit from a \meta{Class} reflecting
the derived class in the inheritance:

\begin{minted}{cpp}
template <>
struct scope<MetaInheritance>
 : MetaClass
{ };
\end{minted}

\subsubsection{\texttt{access\_specifier}}

A template struct \verb@access_specifier@ should be defined and should inherit from
a \meta{Specifier} reflecting one of the \verb@private@, \verb@protected@ and
\verb@public@ access specifiers.

\begin{minted}{cpp}
template <typename T>
struct access_specifier;

template <>
struct access_specifier<MetaInheritance>
 : MetaSpecifier
{ };
\end{minted}

\subsubsection{\texttt{inheritance\_specifier}}

A template struct \verb@inheritance_specifier@ should be defined and should inherit from
a \meta{Specifier} reflecting one of the \verb@virtual@ and "none" access specifiers.

\begin{minted}{cpp}
template <typename T>
struct inheritance_specifier;

template <>
struct inheritance_specifier<MetaInheritance>
 : MetaSpecifier
{ };
\end{minted}

\subsubsection{\texttt{base\_class}}

A template struct \verb@base_class@ should be defined and should inherit from
a \meta{Class} reflecting the base class in the inheritance:

\begin{minted}{cpp}
template <typename T>
struct base_class;

template <>
struct base_class<MetaInheritance>
 : MetaClass
{ };
\end{minted}

