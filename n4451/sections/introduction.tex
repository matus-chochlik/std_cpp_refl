\section{Introduction}

Reflection and reflective programming can be used
for a wide range of tasks, such as the implementation
of serialization-like operations, object-relational mapping,
remote procedure calls, scripting, automated GUI-generation,
implementation of several software design patterns, etc.

However, C++ as one of the most prevalent programming languages 
still lacks a standardized reflection facility.

In this paper we propose to add native support for
compile-time reflection to C++ by the means of compiler generated
types providing basic metadata describing various program constructs.
These metaobjects, together with some additions to the standard
library can later be used to implement other third-party libraries
providing both compile-time and run-time high-level
reflection utilities.

When finalized, the reflection facility providing compile-tim metadata
should be as complete as possible in order to be applicable in a wide
range of scenarios and to allow to implement custom higher-level
static and dynamic reflection software libraries and reflection-based
utilities.

However we recognize that implementing the whole proposed set of metaobjects
all at once may be problematic and that a more gradual approach is necessary.
Therefore\footnote{And to keep the number of pages of the main body of the proposal reasonable.}
in the main body of this paper we propose to add in the first phase
only a subset of the metaobjects from N4111, which we assume to be essential
and which provide a good starting point for future extensions.

The appendices still contain the whole set of metaobjects including their
full interfaces as currently envisioned\footnote{As previously described in N4111 with several
important changes}. The appendices are provided only to keep the bigger picture
in mind, but are not part of the current proposal.

