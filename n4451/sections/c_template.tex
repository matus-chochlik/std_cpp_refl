\subsection{MetaTemplate}
\label{concept-MetaTemplate}

\meta{Template} is a \meta{NamedScoped} and either a \meta{Class} or a \meta{Function}.

\textbf{Note:} The \meta{Template} concept slightly modifies the requirements
of the \meta{Class} and \meta{Function} concepts.

In addition to the requirements inherited from \meta{NamedScoped},
models of \meta{Template} must conform to the following:

The \verb@is_template@ template class specialization for a \meta{Template} should
inherit from \verb@true_type@:

\begin{minted}{cpp}
template <>
struct is_template<MetaTemplate>
 : true_type
{ };
\end{minted}

\subsubsection{\texttt{template\_parameters}}

A template class called \verb@template_parameters@ should be defined and should
inherit from a \concept{MetaobjectSequence} of \meta{TemplateParameter}s,
reflecting the individual type or constant template parameters:

\begin{minted}{cpp}
template <typename T>
struct template_parameters;

template <>
struct template_parameters<MetaTemplate>
 : MetaobjectSequence<MetaTemplateParameter>
{ };
\end{minted}

\subsubsection{\texttt{instantiation}}

A template class \verb@instantiation@ should be defined and should
inherit from a \meta{Instantiation} reflecting a concrete instantiation of
the template reflected by this \meta{Template}:

\begin{minted}{cpp}
template <typename T, typename ... P>
struct instantiation;

template <typename ... P>
struct instantiation<MetaTemplate, P...>
 : MetaInstantiation
{ };
\end{minted}

For example if \verb@__meta_std_pair@ is a \meta{Template} and a \meta{Class} reflecting
the \verb@std::pair@ template and \verb@__meta_std_pair_int_double@ is a \meta{Instantiation}
and a \meta{Class} reflecting the \verb@std::pair<int, double>@ class then:

\begin{minted}{cpp}
static_assert(
	is_base_of<
		__meta_std_pair_int_double,
		instantiation<__meta_std_pair, int, double>
	>() , ""
);
\end{minted}
