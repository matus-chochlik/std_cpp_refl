\subsection{MetaTypedef}
\label{concept-MetaTypedef}

\meta{Typedef} is a \meta{Type} reflecting \verb@typedef@s.

In addition to the requirements inherited from \meta{Type}, the following is required:

The \verb@metaobject_category@ template class specialization for a \meta{Typedef} should
inherit from \verb@typedef_tag@:

\begin{minted}{cpp}
template <>
struct metaobject_category<MetaTypedef>
 : typedef_tag
{ };
\end{minted}

\subsubsection{\texttt{type}}

A template class called \verb@type@ should be defined and should inherit from the \meta{Type}
reflecting the "source" type of the typedef:

\begin{minted}{cpp}
template <typename T>
struct type;

template <>
struct type<MetaTypedef>
 : MetaType
{ };
\end{minted}

For example if \verb@__meta_std_string@ is a \meta{Typedef} reflecting the \verb@std::string@
typedef and \verb@__meta_std_basic_string_char@ is the \meta{Type} that reflects
the \verb@std::basic_string<char>@ type, and \verb@std::string@ is defined as:

\begin{minted}{cpp}
namespace std {
typedef basic_string<char> string;
}
\end{minted}

then the specialization of \verb@type@ for \verb@__meta_std_string@ should be following:

\begin{minted}{cpp}
template <>
struct type<__meta_std_string>
 : __meta_std_basic_string_char
{ };
\end{minted}

