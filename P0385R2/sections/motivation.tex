\section{Motivation}

Generic programming and metaprogramming supported by reflection can be valuable
tools in the implementation of an extensive range of various use cases or
programming patterns, including but not limited to:

\begin{itemize}

\item serialization or conversion of data from the native C++ representation
into a standard or custom, text-based or binary format like XML, JSON, XDR, ASN1,
etc.,

\item (re-)construction of instances of both \say{atomic} and structured types
from external data representations, like those listed above, or
from the data stored in a relational database, or from data entered by
a user through a user interface, or queried through a web service API,

\item automated implementation of comparison operations or hash functions
for structured types,

\item automated generation of user interface elements,

\item automatic generation of a relational schema from the application
object model and object-relational mapping (ORM),

\item support for scripting or implementation of input data parsers,

\item support for remote procedure calls (RPC) / remote method invocation (RMI),

\item inspection and manipulation of existing objects via a user interface
or a web service,

\item visualization of objects or structured data and their relationships,

\item automatic or semi-automatic implementation of certain software design
patterns, for example the factory pattern,

\item implementation of cross-cutting aspects like debugging, logging, profiling,
access control, etc.,

\item implementation of source code generators.

\end{itemize}

Some of the use cases listed above are described in more detail in section
\ref{use-cases-examples}.

There are several approaches to the implementation of the mentioned functionality.
The most basic, straightforward and also usually the most
error-prone is manual implementation. Many of the tasks listed above
are inherently repetitive and basically require to process and organize
programming language elements\footnote{types, structures, containers, functions,
constructors, class member variables, enumerated values, etc.}
in a very uniform way which could be transcribed into a metaprogram\footnote{with
varying level of complexity}.

This leads to the second, heavily used approach: preprocessing
and parsing of the program source text by a usually very specific external
program like, a documentation generation tool, an interface definition language
compiler for a RPC/RMI framework, a web service interface generator,
a rapid application development environment with a form designer, etc.,
resulting in additional program source code, which is then integrated into
the project and compiled into the final application binary.

This approach has several problems. First, it requires the external
tools which may not fit well into the build system or may not be portable
between platforms or be free; second, such tools are task-specific
and many of them allow only a limited, if any, customization of the output
and third, there is a lot of repeated code related to the parsing, the
representation and the manipulation of the input program source.

Another way to automate these tasks is to use reflection and metaprogramming.
Metaprogramming is the tool for transforming one program into another based
on some meta-algorithm and reflection provides the input data for that algorithm
directly from the compiler without the need for an external source code parser.

For example if we want to log the execution of a function, reflection may
be used as a source of metadata:

\begin{minted}[tabsize=4]{cpp}
template <typename T>
T min(const T& a, const T& b)
{
	log()   << "function: min<"
	        << get_base_name_v<get_aliased_t<$reflect(T)>>
	        << ">("
	        << get_base_name_v<$reflect(a)> << ": "
	        << get_base_name_v<get_aliased_t<get_type_t<$reflect(a)>>>
	        << " = " << a
	        << get_base_name_v<$reflect(b)> << ": "
	        << get_base_name_v<get_aliased_t<get_type_t<$reflect(b)>>>
	        << " = " << b
	        << ")" << std::endl

	return a<b?a:b;
}
\end{minted}

Calling the \verb@min@ function:

\begin{minted}[tabsize=4]{cpp}
double m = min(12.34, 23.45);
\end{minted}

would produce the following log entry:

\begin{verbatim}
function: min<double>(a: double = 12.34, b: double = 23.45)
\end{verbatim}

