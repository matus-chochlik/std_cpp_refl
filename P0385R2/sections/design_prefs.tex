\subsection{Design considerations}
\label{design-prefs}

The proposed static reflection facility has been designed with the following
considerations and goals in mind. 
Note that some of the principles listed here
apply only to the whole reflection facility as it is envisioned
to look in the future, not to the initial, limited subset from P0194R2.

\subsubsection{Completeness and reusability}
\label{design-completeness}

The metadata provided by reflection is reusable in many situations
and for many different purposes.
It does not focus on nor is limited only to the simple and immediately obvious
use cases. New use cases which we are not aware of at this moment, may emerge
in the future. So having or not having a compelling use case for a particular
feature is a factor in the decision whether to include it, but it should
not be the most important one.

When completed, the proposed reflection facility will provide as much useful
metadata as possible, reflecting various base-level
declarations like types, namespaces, variables, functions, templates,
specifiers\footnote{Like constness, storage-class, access, etc. specifiers} and
will provide access to scope members, base classes, etc.

This will make the compiler-assisted reflection\footnote{Either by itself or
serving as the foundation for other standard or third-party libraries.} a useful
tool in a wide range of scenarios during both compile-time
and run-time and under various paradigms\footnote{Object-oriented, functional,
etc.} depending on the application needs.

\subsubsection{Consistency}
\label{design-consistency}

The reflection facility as a whole is consistent, instead of being
composed of several ad-hoc, individually-designed parts. This makes
its interface more tidy, coherent and easier to learn and teach.

\subsubsection{Encapsulation}
\label{design-encapsulation}

The metadata is not exposed directly to the user by many different compiler
built-ins, operators or special expressions.
Instead it is accessible through conceptually well-defined interfaces,
inspired by the existing {\em type-traits}, already present in
the C++ standard template library.

\subsubsection{Stratification}
\label{design-stratification}

Reflection is non-intrusive and the metaobjects are separated
from the base-level language declarations which they reflect:

\begin{minted}[tabsize=4]{cpp}
float foo = 42.f;

using m_std = $reflect(std);
using m_int = $reflect(int);
using m_foo = $reflect(foo);
\end{minted}

instead of for example

\begin{minted}[tabsize=4]{cpp}
using m_std = std::reflect();
using m_int = int::reflect();
using m_foo = foo::reflect();
\end{minted}

This is achieved by using the reflection operator which hides most of the
\say{magic}, isolates reflection from the rest of the language,
for example by allowing to pass expressions not valid elsewhere as operands.

It is also consistent with the syntax of other existing operators in C++:

\begin{minted}[tabsize=4]{cpp}
float foo = 42.f;

using foo_type = decltype(foo);
using meta_foo = $reflect(foo);
\end{minted}

\subsubsection{Meta-level reification}
\label{design-reification}

The metaobjects giving a first-class identity to declarations
which are only second-class in base-level C++,
allow to partially \say{reify} namespaces, specifiers, etc., without actually
making them first-class citizens. This in turn allows to pass their reflections around
metaprograms even if it was not possible with the base-level declarations.

\begin{minted}[tabsize=4]{cpp}
template <typename X>
struct my_meta_algorithm
{
	using result = /* ... */;
};

using x = my_meta_algorithm<$reflect(std)>::result; // OK

using y = my_meta_algorithm<std>::result; // Error

\end{minted}

\subsubsection{Ontological correspondence}
\label{design-onto-corr}

The meta-level 
corresponds to the ontology of the base-level C++ language
which it reflects. This basically means that all important existing language
features\footnote{Within reason, we certainly do not want to reflect every token
in a C++ program.} will eventually be reflected by appropriate metaobjects,
but new ones not having an equivalent in the base-level language\footnote{At least
conceptually.} will not be invented.

Ontological correspondence is one of the main factors driving the definition of the
individual metaobject concepts and the design of their interfaces.


\subsubsection{Efficiency}
\label{design-efficiency}

The proposed reflection is fine grained as much as possible.
Things that are not needed for a particular application,
are not compiled into its program code nor result in
increased compiler footprint or compilation times.

The proposed reflection facility makes a completely lazy implementation of
metaobjects possible. Metaobjects are created only when requested and
the reflection operator is able to generate very
lightweight types providing internal links back to the reflected declaration.
The actual
metadata\footnote{Like a compile-time string containing the identifier,
the list of metaobjects reflecting class members or the scope of a declaration,
etc.} is materialized only when requested by the programmer via the
templates which act as the metaobject interface.

\subsubsection{Ease of use}
\label{design-ease-of-use}

Although reflection-based metaprogramming should allow to implement very
complicated meta-algorithms, we try to adhere to the principle that things
should be kept as simple as possible, but not simpler\footnote{Credits to
whoever said that.}.

This can be achieved by having a solid and powerful compiler-assisted reflection
as the foundation and by implementing a simplifying facade on top of it once
the common use-cases are identified.

Utilities such as {\em enum-to-string}, {\em string-to-enum} or {\em get-type-name},
etc., can be implemented on top of reflection instead of each of them using their own
\say{dedicated compiler magic}.

\subsubsection{Integration}
\label{design-integration}

The proposed reflection facility is easily
usable with the existing introspection utilities\footnote{Like the
{\em type-traits} or \texttt{typeid} and \texttt{std::type\_info}.}
already provided by the standard library and by other third-party libraries.

For example:

\begin{minted}{cpp}
using meta_int = $reflect(int);
static_assert(is_integral_v<reflect::get_reflected_type_t<meta_int>>, "");
\end{minted}

\subsubsection{Extensibility}
\label{design-extensibility}

It is important be able to gradually add new features and to allow
reflecting new declaration kinds in the future without introducing breaking changes.
The metaobjects make this goal easily achievable.

If we want to add reflection of previously unsupported declaration kind, for example
the reflection of functions, we define a new {\em metaobject concept} like
\meta{Function} and \meta{OverloadedFunction}.

\begin{minted}{cpp}
namespace reflect {
\end{minted}
\begin{minted}[xleftmargin=1em,tabsize=4]{cpp}

template <Object T>
concept bool Function = 
	Named<T> && ScopeMember<T> && Typed<T>
	__metaobject_is_meta_function(T);

template <Object T>
concept bool OverloadedFunction =
	Named<T> && ScopeMember<T> &&
	__metaobject_is_meta_overloaded_function(T);

\end{minted}
\begin{minted}{cpp}
} // namespace reflect
\end{minted}

Then it is necessary to extend the list of expressions which can appear
as operands to \verb@$reflect@ to include specifiers, so that the following
is valid code:

\begin{minted}{cpp}
using meta_func = $reflect(std::time);
\end{minted}

To extend the interface of an existing metaobject by adding a new operation
returning for example the elaborated type specifier of a class we need to
add a new template like:

\begin{minted}{cpp}
namespace reflect {
\end{minted}
\begin{minted}[xleftmargin=1em,tabsize=4]{cpp}
template <Class T>
struct get_member_functions
{
	using type = /* unspecified-metaobject */;
};

template <OverloadedFunction T>
struct get_overloads
{
	using type = /* unspecified-metaobject */;
};

template <Function T>
struct get_parameters
{
	using type = /* unspecified-metaobject */;
};

\end{minted}
\begin{minted}{cpp}
} // namespace reflect
\end{minted}

