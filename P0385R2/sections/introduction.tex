\section{Introduction}

This paper accompanies the P0194R3 paper which we want to keep brief, technical
and to the point and which will eventually result in the final wording to be
included into the standard, if this proposal is accepted. There is also
a short introductory paper -- P0578 which gives a concise introduction
and overview of our reflection proposal.

We are writing this paper with several goals in mind:

\begin{itemize}
\item To define and explain the reflection-related terminology.
\item To keep a written record of the rationale behind the design of the
proposed static reflection facility.
\item To keep the answers to the frequently asked questions about
the decisions we've made in this proposal in one place so that we can avoid
having to write them over and over from scratch in various discussions\footnote
{Also to help us remember what the answers were.}.
\item To enumerate and describe the use cases for the various features
which we included in the proposal.
\item To provide concrete examples of usage.
\item To discuss the possibilities of the evolution of reflection in the future.
\end{itemize}

The text of this paper includes several revised, updated and extended parts
from the previous papers: N3996, N4111, N4451, N4452 and previous revisions
of P0385.

There is also an experimental implementation of the P0194R2 proposal
described in greater detail in section~\ref{implementation}
and the {\em Mirror} reflection library built on top of the proposed
static reflection facility described in section~\ref{mirror-lib}.

\subsection{Terminology}

In order to avoid confusion about terminology, this section provides definitions
for several important terms used throughout the text of the paper and
explains what we mean when using them in the context of this paper.

\subsubsection{Base-level, meta-level}
\label{term-base-meta-level}

When speaking generally, the {\em meta-level} is some higher level of abstraction
conceptually describing a lower, {\em base-level} which is the primary subject
of our endeavors.

In the context of this paper the base-level is the structure of a C++ program.
The meta-level is an abstraction partially describing that structure,
mainly the declarations of the program.

\subsubsection{Metadata}
\label{term-metadata}

{\em Metadata} is generally a piece of data conceptually describing some other,
{\em primary} data.

In the context of this paper, metadata is data providing information
about the base-level structure of a program.

{\em Static metadata} is metadata which can be manipulated or reasoned
about at compile-time by the compiler.

The metadata itself has its own structure.
For example metadata describing the base-level declaration of a class 
from a C++ program includes;
\begin{itemize}
\item the name of the class,
\item its scope,
\item list of its base classes,
\item list of data members,
\item list of nested types like type aliases, classes and enumerations,
\item the class key specifier, the access specifier,
\item source location information,
\item etc.
\end{itemize}

\subsubsection{Metaprogramming}

{\em Metaprogramming} is a kind of programming with the ability to treat and
manipulate other programs as data, so both the input and the output of
a metaprogram is usually a program. The language in which metaprograms
are written is called the {\em metalanguage} and it can be a different or the
same language as the one used to write the primary program.

C++ metaprogramming can be done both in an external language\footnote{For example a
C++ code generator written in Python or Bash.}, in a C++ compiler plug-in,
or in C++ itself. 

In C++ metaprogramming usually takes the form of an external
{\em source code generator}\footnote{Like Qt's MOC.}, or the form
of {\em preprocessor}, {\em template} or {\em constexpr} metaprogramming.

With template metaprogramming we use C++'s type system
as a standalone functional programming language \say{interpreted} by a C++
compiler, with \say{variables} being
represented by types (or compile-time constants),
\say{data structures} by instantiations of templates,
\say{subroutines} by class templates or template aliases, and
algorithms or \say{programs} by compositions of the above.

Unless stated otherwise when we say \say{metaprogramming} in the following text,
we mean template metaprogramming\footnote{Although we envision that
a constexpr-metaprogramming interface will also be added later in the future.}.

\subsubsection{Reflection}

In the context of computer science,
the term {\em reflection} refers to the ability of a program to examine
and possibly modify its own structure and/or behavior.

When combined with metaprogramming, this can include modification of the existing
or the definition of new data structures, doing changes to algorithms or changing the way
a program code is interpreted\footnote{Mostly in interpreted languages.}.

For the purpose of this paper -- {\em reflection is primarily the process
of obtaining metadata, but it is also a mechanism for code generation}.

In the future the meaning can be expanded to include modification of the program
in ways exceeding the capabilities of current template
metaprogramming\footnote{For example generating new class data members.}.

\subsubsection{First-class object, second-class object}
\label{term-first-class}

First-class objects are also known as first-class {\em citizens}, {\em types},
{\em entities} or {\em values}.
In the context of programming language design they describe an entity that satisfies
the following:

\begin{itemize}
	\item{Can be stored in a named variable or a data structure.}
	\item{Can be passed as an argument to a subroutine.}
	\item{Can be returned as a result of a subroutine.}
	\item{Has an intrinsic identity making the entity unique and distinguishable.}
\end{itemize}

Since this paper deals with compile-time static reflection and its use
in compile-time metaprogramming, we will be talking about first- or second-class
citizens in this regard. 
For the purpose of this paper a first-class object is something that
we can distinguish and reason about at compile-time and what can be passed around
as \say{data} in metaprograms -- something that can be a template parameter.

This means that a type, a template instance or a compile-time integral constant
is for our purposes a first-class object:

\begin{minted}{cpp}
// compile-time "values"
struct value_a { };
struct value_b { };

// a compile-time "subroutine"
template <typename Param>
struct identity {
	using result = Param;
};

// "values" are equality comparable and they are distinguishable
assert(!std::is_same_v<value_a, value_b>);
// and can be used for overloading
assert(!std::is_same_v<identity<value_a>, identity<value_b>>);

// they can be stored in named "variables" ...
using x = value_a;

// ... while maintaining their identity
assert(std::is_same_v<value_a, x>);

// they can be passed as parameters to subroutines
identity<value_a>;
identity<value_b>;

// they can be returned from subroutines
using y = identity<value_a>::result;
using z = identity<value_b>::result;

// ... while still maintaining their identity
assert(std::is_same_v<value_a, y>);
assert(std::is_same_v<value_b, z>);
// and still being distinguishable
assert(!std::is_same_v<y, z>);
\end{minted}


On the other hand a namespace, a type alias or a template parameter do not have
some or any of these properties.

\begin{minted}{cpp}
// other entities
namespace std { }
namespace foo = std;
using bar = unsigned;
using baz = unsigned;

template <typename Param>
struct identity {
	using result = Param;
};

// they are not distinguishable
std::is_same_v<bar, baz>;
// and cannot be used for overloading
std::is_same_v<identity<bar>, identity<baz>>;

// they are not even comparable
std::is_same_v<std, foo>;

// they cannot be passed as arguments to subroutines
identity<std>;
identity<foo>;

// nor returned as a result
namespace y = identity<foo>::result;

// those which can be returned ...
using z = identity<bar>::result;
// ... do not maintain their unique identity
std::is_same_v<z, baz>;
\end{minted}

From the above follows, that a {\em second-class object} is everything else that
is not a first-class object, for example a namespace or a typedef.

\subsubsection{Reification}
\label{term-reification}

Generally speaking {\em reification} or \say{thingification}\footnote{From latin
\say{rei}, the dative form of \say{res} -- a thing.}, is making something
real, bringing it into being as an entity with its own identity,
or making something concrete.

In regard to programming languages, reification is often defined as making
an entity in the language a first-class object. So in the context of C++ template
metaprogramming a type is reified, but a namespace or a specifier is not.

