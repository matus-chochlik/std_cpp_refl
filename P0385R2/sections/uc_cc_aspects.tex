\subsection{Cross-cutting aspects}

We need to execute the same action or a set of unrelated actions at the entry of
or at the exit from the body of each function from a set of multiple functions
meeting some criteria every time one of them is called.

The actions may be related to logging, debugging, profiling, but also to access
control\footnote{Not allowing users to execute operations for which they
are not authorized.}, etc.

The condition which selects the functions for which the action is invoked might
be something like:
\begin{itemize}
\item each member function of a particular class,
\item each function defined in some namespace,
\item each function returning values of a particular type or having
	a particular set of parameters,
\item each function whose name matches a pattern,
\item each function declared in a particular source file,
\item and so on and various combinations of the above.
\end{itemize}

It may not be possible to tell in advance the relations between the aspects
and the individual functions or these relations may vary for different builds
or build configurations.
Furthermore we want to be able to quickly change the assignment of actions to
functions in one place instead of going through the whole project source which
may consists of dozens or even hundreds of files.

We want for example temporarily enable logging of the entry and exit of each member
function of class \verb@foo@,
or we need to count the number of invocations of functions defined in the \verb@bar@
namespace with names not starting with an underscore, or we want to throw
the \verb@not_logged_in@ exception at the entry of each member function
of class \verb@secure@ if the global function \verb@user_logged_in@ returns \verb@false@.

Without reflection something like this could be implemented in the following way:

\begin{minted}[tabsize=4]{cpp}

class logging_aspect
{
public:
	template <typename ... P>
	logging_aspect(const char* func_name, P&&...)
	{
		// log entry to std::clog
	}

	~logging_aspect(void)
	{
		// log exit
	}
};

class profiling_aspect
{
	/* ... */
};

class authorization_aspect
{
public:
	template <typename ... P>
	authorization_aspect(const char* func_name, P&&...)
	{
		if(contains(func_name, "secure"))
		{
			if(!::is_user_logged_in())
			{
				throw not_logged_in(func_name);
			}
		}
	}
};

template <typename RV, typename ... P>
class func_aspects
 : logging_aspect
 , profiling_aspect
 , authorization_aspect
/* ... etc. ... */
{
public:
	func_aspects(
		const char* name,
		const char* file,
		unsigned line,
		P&&... args
	): logging_aspect(name, file, line, args...)
	 , profiling_aspect(name, file, line, args...)
	 , authorization_aspect(name, file, line, argc...)
	/* ... etc. ... */
	{ }
};

template <typename RV, typename ... P>
func_aspects<RV, P...>
make_func_aspects(
	const char* name,
	const char* file,
	unsigned line,
	P&&...args
);


void func1(int a, int b)
{
	const auto _fa = make_func_aspects<void>(
		__func__,
		__FILE__,
		__LINE__,
		a, b
	);
	/* function body */
}

double func2(double a, float b, long c)
{
	const auto _fa = make_func_aspects<double>(
		__func__,
		__FILE__,
		__LINE__,
		a, b, c
	);
	/* function body */
}

namespace foo {

long func3(int x)
{
	const auto _fa = make_func_aspects<long>(
		__func__,
		__FILE__,
		__LINE__,
		x
	);
	/* function body */
}

} // namespace foo
\end{minted}

Obviously this is very repetitive and it can get quite tedious and error-prone
to supply all this information to the aspects in each function manually.
Also if the signature or the name of the function changes the construction
of the \verb@func_aspects@ instance must be updated accordingly.
With the help of reflection things can be simplified considerably:

\begin{minted}[tabsize=4]{cpp}

template <typename MetaFunction, typename Enabled>
class logging_aspect_impl;

template <typename MetaFunction>
class logging_aspect_impl<MetaFunction, false_type>
{ };

template <typename MetaFunction>
class logging_aspect_impl<MetaFunction, true_type>
{
public:
	logging_aspect_impl(void)
	{
		clog
			<< get_base_name_v<MetaFunction>
			<< "("
			/* ... */
			<< ")"
			<< endl;
	}
};

template <typename MetaFunction>
constexpr bool logging_enabled =
	is_same_v<
		reflexpr(std),
		get_scope_m<MetaFunction>
	> && is_same_v<
		std::string,
		get_reflected_type_t<get_result_m<MetaFunction>>
	> &&
	/* ... etc. ... */
;

template <typename MetaFunction>
using logging_aspect =
	logging_aspect_impl<
		MetaFunction,
		logging_enabled<MetaFunction>
	>;

template <typename MetaFunction, typename Enabled>
class authorization_aspect_impl;

template <typename MetaFunction>
class authorization_aspect_impl<MetaFunction, false_type>
{ };

template <typename MetaFunction>
class authorization_aspect_impl<MetaFunction, true_type>
{
public:
	authorization_aspect_impl(void)
	{
		if(!::is_user_logged_in())
		{
			throw not_authorized(
				full_name<MetaFunction>()
			);
		}
	}
};

template <typename MetaFunction>
struct autorization_enabled
 : integral_constant<
	bool,
	is_base_of<
		reflexpr(foo::bar),
		get_scope_m<MetaFunction>
	> && constexpr_starts_with(
		get_base_name_v<MetaFunction>,
		"secure_"
	) &&
	/* ... etc. ... */
;

template <typename MetaFunction>
using authorization_aspect =
	authorization_aspect_impl<
		MetaFunction,
		typename authorization_enabled<MetaFunction>::type
	>;

template <typename MetaFunction>
class func_aspects
 : logging_aspect<MetaFunction>
 , profiling_aspect<MetaFunction>
 , authorization_aspect<MetaFunction>
/* ... etc. ... */
{
public:
};

void func1(int a, int b)
{
	const func_aspects<reflexpr(this::function)> _fa;
	/* function body */
}

double func2(double a, float b, long c)
{
	const func_aspects<reflexpr(this::function)> _fa;
	/* function body */
}

namespace foo {

long func3(int x)
{
	const func_aspects<reflexpr(this::function)> _fa;
	/* function body */
}

} // namespace foo

\end{minted}

In this case the same expression is used in all functions
regardless of their name and signature and the aspects get all the information
they require from the metaobject reflecting the function. All the data
obtained from the metaobjects is available at compile-time so various
specializations of the aspect classes can be implemented as required.

This same technique could also be used with instances of classes:

\begin{minted}[tabsize=4]{cpp}

template <typename MetaClass>
class class_aspects
 : logging_aspects<MetaClass>
/* ... etc. ... */
{
public:
	class_aspects(get_reflected_type_t<MetaClass>* that);
};

class cls1
{
private:
	int member1;
	/* ... other members ... */
	class_aspect<reflexpr(this::class)> _ca;
public:
	cls1(void)
	 : member1(...)
	 , _ca(this)
	{ }
};

\end{minted}

Class aspects like these could also be used for logging, monitoring of object
instantiation, resource leak detection, etc.

