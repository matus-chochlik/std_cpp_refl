\subsection{Reversing reflection}
\label{fut-reverse-reflection}

The \verb@$reflect@ operator allows to obtain the meta-level representation
of a base-level declaration. But what if we wanted to do the opposite: to
get back the original declaration from a metaobject?

This of course would make sense only for a subset of metaobjects, so let's
define a new concept like \meta{Reversible}.

One way to achieve this would be to add new operator -- \verb@$unreflect@.

Let's look at some possible use cases for this feature.
The most trivial one is to replace the \texttt{get\_reflected\_type}
template from the interface of \meta{Type}:

\begin{minted}{cpp}
using MT = $reflect(int);

// ... lots of other code here ...

// Now we arrived to a point where we need
// to get the base-level type reflected by `MT`.

// Instead of using get_reflected_type ...
using T = get_reflected_type_t<MT>;

// we could do:
static_assert(reflect::Reversible<MT> && reflect::Type<MT>);
using T = $unreflect(MT);
static_assert(is_same_v<T, int>);

\end{minted}

But that's far from being the only use case. If we could get back to
the original namespace or variable;

\begin{minted}[tabsize=4]{cpp}
namespace foo {
	std::string str;

	void bar(const std::string&);
} // namespace foo

using MV = $reflect(foo::str); // Meta-Variable

// ... lots of code here ...
// using namespace foo;
using namespace $unreflect(get_scope_t<MV>);

// [foo::]bar(foo::str);
bar($unreflect(MV));

\end{minted}

or even selecting whole namespaces (meta-)programmatically;

\begin{minted}[tabsize=4]{cpp}

namespace foo {
	void func1(const std::string&);
	void func2(int, int, int);
	int func3(long, short, bool);
} // namespace foo


namespace bar {
	void func1(const std::string&);
	void func2(int, int, int);
	int func3(long, short, bool);
} // namespace bar

template <typename MN>
void algorithm(const string& str, int i, int j) {
	// [foo|bar]::func1
	$unreflect(MN)::func1(str);

	// [foo|bar]::func2([foo|bar]::func3(i*j, 64, i == j), i, j);
	$unreflect(MN)::func2($unreflect(MN)::func3(i*j, 64, i == j), i, j);
}

void func(const string& str, int i, int j, bool want_foo) {
	if(want_foo) {
		algorithm<$reflect(foo)>(str, i, j);
	} else {
		algorithm<$reflect(bar)>(str, i, j);
	}
}

\end{minted}

or get back to the original function;

\begin{minted}[tabsize=4]{cpp}

namespace foo {
	void bar(const std::string&);
} // namespace foo

using MF = $reflect(foo::bar); // Meta-Function

// ... lots of code here ...
//   foo::bar("hello");
$unreflect(MF)("hello");

\end{minted}

or back to the original class data member;

\begin{minted}[tabsize=4]{cpp}

struct my_struct {
	int i;
	float f;
};

// Meta-DataMember
using MDM = $reflect(foo::bar::i);

my_struct x {123, 45.67f};

assert(x.i == 123);
// x.i = 234;
x.$unreflect(MDM) = 234;
assert(x.i == 234);

\end{minted}

or go back to the original template;

\begin{minted}{cpp}

// Meta-Template
using MTpl = $reflect(std::pair);

// std::pair<int, std::string> p;
$unreflect(MTpl)<int, std::string> p{10, "Hi!"};

\end{minted}

and so on.

Combined with the fact that all metaobjects are first-class
entities this would make a very powerful feature, which would essentially
allow to \hyperref[term-reification]{reify} all reflectable declarations at
the meta-level, even those which are
second-class without actually reifying them at the base-level\footnote
{At the cost of the round-trip through the meta-level.}

