\subsection{Design evaluation}

\subsubsection{The good}

The proposed reflection facility\footnote{assuming it is fully completed}:

\begin{itemize}
\item covers many different use cases,
\item is fairly powerful and expressive,
\item is non-intrusive,
\item is fine-grained,
\item allows for efficient implementations,
\item allows to manipulate and reason about all provided metadata at compile-time,
\item gives the metadata a structure by arranging it into metaobjects,
\item makes the metaobjects first-class entities allowing to pass
	representations of second-class base-level language entities
	around metaprograms as arguments and return values and store them
	in named \say{variables}\footnote{We cannot stress the importance of this
	feature enough.},
\item can serve as the foundation for other, compile-time or run-time reflection
	utilities implementing other interfaces or fa\c{c}ades aimed at various
	paradigms or use cases,
\item contains and isolates all the required changes within the reflection operator,
\item limits the impact on existing code by adding only a single reserved keyword,
\item does not require any other changes to the core language, especially no new rules
	for template parameters.
\end{itemize}

\subsubsection{The bad}

This proposal requires the addition of a new operator -- \texttt{\$reflect}
which {\em may} cause conflicts with identifiers in existing code.

For what it's worth, we have performed a quick analysis on \num{994} third-party,
open-source repositories of C++ projects, hosted on
\url{http://github.com/}\footnote{The main branches of original repositories,
not forks.}, where we counted identifiers in the C++ source files.

We have found \num{646313149} instances of \num{7903042} {\em distinct} words
matching the C++ identifier rules.  We did not find {\em any} occurrence of
\say{\$reflect}.

\subsubsection{The ugly}

The complexity of the proposal\footnote{stemming from the complexity of the
base-level language} makes it too verbose for certain simple use cases
or it may be difficult to learn for beginners.

On the other hand we do not want to trade its usefulness for \say{simplicity}
and as already said simplifying wrappers aimed at trivial use cases can 
and {\em will} be devised and added to the standard library.
See also the discussion \hyperref[faq-hard-on-novices]{in the FAQ section}.

Representing the metaobjects as types may be suboptimal in some simple
use cases. For example:

\begin{minted}[tabsize=4]{cpp}
std::string(std::reflect::get_base_name_v<$reflect(T)>);
\end{minted}

However, the compilers already are doing many other AST transformations,
optimizations and elisions, and their developers have many tricks at their
disposal to make the above as efficient as say:

\begin{minted}[tabsize=4]{cpp}
std::string(operator_get_base_name_of(T));
\end{minted}

