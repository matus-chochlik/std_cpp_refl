\subsection{Design evaluation}

The proposed reflection facility -- assuming it is fully completed -- will have
the following properties:

\begin{itemize}
\item covers many different use cases,
\item is fairly powerful and expressive,
\item is non-intrusive,
\item is fine-grained,
\item allows for efficient implementations,
\item allows to manipulate and reason about all provided metadata at compile-time,
\item gives the metadata a structure by arranging it into metaobjects,
\item makes the metaobjects first-class entities allowing to pass
	representations of second-class base-level language entities
	around metaprograms as arguments and return values and store them
	in named \say{variables}\footnote{We cannot stress the importance of this
	feature enough.},
\item it can serve as the foundation for other, compile-time or run-time reflection
	utilities implementing other interfaces or fa\c{c}ades aimed at various
	paradigms or use cases,
\item contains and isolates all the required changes within the reflection
	operator(s),
\item limits the impact on existing code by prefixing the names of new operators
	with the \verb@$@ sign and by putting the concepts and templates into
	the nested \verb@std::reflect@ namespace,
\item does not require any other changes to the core language, especially no
	new rules for template parameters.
\end{itemize}

