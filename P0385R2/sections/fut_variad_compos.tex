\subsection{Variadic composition}
\label{fut-variadic-composition}

Another feature which\footnote{together with the ability to generate identifiers}
has the potential to greatly simplify the implementation of several important
use cases, including the \hyperref[use-case-struct-transf]{structure data
member transformations}, is {\em variadic composition}.

We already have variadic {\em inheritance} that is used in the implementation
of tuples for example, but we don't have variadic {\em composition}.

The reason for this is simple -- every class data member needs to have a name
which is unique in the scope of its class and since we don't have the ability
to generate identifiers, we cannot implement variadic composition.

For instance, the structure-of-arrays transformation described in section
\ref{use-case-struct-transf} could be simplified by using variadic composition,
together with the \say{identifier formatting operator} described in section
\ref{fut-ident-formatting} and the \verb@unpack_sequence@ helper template
described in section \ref{fac-unpack-sequence}:

\begin{minted}[tabsize=4]{cpp}
template <DataMember ... MDM>
struct soa_helper
{
	std::vector<reflect::get_reflected_type_t<reflect::get_type_t<MDM>>>
	$identifier("vec_%1", MDM)...;
};

template <typename T>
using soa = reflect::unpack_sequence_t<
	reflect::get_data_members_t<$reflect(T)>,
	soa_helper
>;
\end{minted}

With the \say{reversible reflection} feature described in section
\ref{fut-reverse-reflection}, the \verb@soa_helper@ could be implemented as:

\begin{minted}[tabsize=4]{cpp}
template <DataMember ... MDM>
struct soa_helper
{
	std::vector<$unreflect(reflect::get_type_t<MDM>)>
	$identifier("vec_%1", MDM)...;
};
\end{minted}
