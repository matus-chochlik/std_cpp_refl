\section{Issues}

\subsection{Interaction with attributes}

\textbf{Q:} {\em Should we even reflect attributes? If so, how will the reflection
of generalized attributes look like?}

Currently generalized attributes are used {\em mostly} as hints to the compiler.
They can help the compiler to do some optimizations\footnote{for example the
\texttt{probably(true)} or \texttt{carries\_dependency} attributes} or to
indicate that the user really means to do something what in other circumstances
could be a bug\footnote{like the \texttt{fallthrough} or \texttt{maybe\_unused}
attributes} that the compiler warns about, etc.

Some people argue that generalized attributes are not supposed to have any
semantic effects -- a compiler should be able to completely strip 
a program of all attributes and it should not have any effects on the behavior
of the program.

Attributes could however be viewed more broadly as a generic mechanism for
annotating declarations\footnote{Associating one or more constant values with
the declarations.} in the code, without explicitly saying that the recipient
can only be the compiler.
There are valid use cases for user annotations and since the annotations
usually provide additional conceptual (meta-)information about a declaration,
reflection is a natural mechanism for accessing the annotations.

We favor the view that on the base-level the compiler should be able to
strip attributes\footnote{or to ignore attributes which it does not understand},
but it should reflect them on the meta-level on demand.

Having covered that, the reflection of the \say{simple} attributes like
\verb@[[attr1]]@ or \verb@[[namespace::attr2]]@ could be pretty straightforward:

We add a \meta{Attribute} concept,
make it conform to \meta{Named} and \meta{Scoped} concepts and then add an
operation to the interface of \meta{Object} like \verb@get_attributes@, returning
a sequence of metaobjects reflecting the attributes on a base-level declaration.

The complication is that attributes can have nearly arbitrary parameters;
\verb@[[probably(true)]]@, \verb@[[deprecated("reason")]]@,
\verb@[[visibility(hidden)]]@, \verb@[[gnu::aligned(64)]]@. The reflection of
such attributes and their arguments is currently unresolved.

\subsection{Interaction with concepts}

\textbf{Q:} {\em Are there any special interactions between compile-time
reflection and concepts?}

Unresolved.

\subsection{Interaction with modules}

\textbf{Q:} {\em Are there any special interactions between compile-time
reflection and modules?}

Unresolved.

