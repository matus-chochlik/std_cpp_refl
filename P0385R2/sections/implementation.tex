\section{Experimental implementation}
\label{implementation}

A fork of the clang compiler with a initial, partial, experimental implementation
of this proposal can be found on github \cite{ReflexprImplementation}.
The modified compiler can be built by following the instructions listed
in \cite{ClangGettingStarted}, but instead of checking out the official clang
repository, the sources on the \texttt{reflexpr} branch of the modified
repository should be used.

The reflection implementation represents the metaobjects as compile-time
constants wrapped in a class template and implements the metaobject operations
with the help of compiler built-in operators taking the metaobject ID values
as arguments. 

It adds a new native integral \verb@__metaobject_id@ type which has the same
bit-width as the \verb@std::uintptr_t@ type on the platform for which
the compiler is built (this allows values of \verb@__metaobject_id@ to represent
pointers in the compiler). Unlike other integral types, \verb@__metaobject_id@
does not have any arithmetic operators and no conversions to other integral types. 

The metaobjects are represented as specializations of the \verb@std::__metaobject@
template: 

\begin{minted}{cpp}
namespace std {

template <__metaobject_id MoId>
struct __metaobject
{
	__metaobject() = default;
	__metaobject(const metaobject&) = default;
};

} // namespace std
\end{minted}

The metaobject ID can be obtained from a metaobject (type) by using the internal
helper template \verb@__unwrap_id@: 

\begin{minted}{cpp}
namespace std {

template <typename T>
struct __unwrap_id;

template <__metaobject_id MoId>
struct __unwrap_id<__metaobject<MoId>>
{
	static constexpr __metaobject_id value = MoId;
};

template <typename T>
constexpr __metaobject_id __unwrap_id_v = __unwrap_id<T>::value;

} // namespace std
\end{minted}

The proposed \verb@std::is_metaobject@ trait can then be implemented as follows: 

\begin{minted}{cpp}
namespace std {

template <typename T>
struct is_metaobject : false_type { };

template <__metaobject_id MoId>
struct is_metaobject<__metaobject<MoId>> : true_type { }

} // namespace std
\end{minted}

The invocation of the \verb@$reflect@ operator yields specializations of
the \verb@std::__metaobject@ template, where the value of the \verb@MoId@
argument is a handle to a representation of the metaobject inside of
the compiler, as described below. 


The metaobject operations and concepts are implemented with the help of compiler
built-ins which take one or several arguments where at least one which is
a metaobject ID (i.e. a compile-time constant of the \verb@__metaobject_id@ type).

These built-ins are then used in the definition of the class templates,
template aliases and concepts which implement the public interface of
the metaobject.

The built-ins can yield compile-time constant values of the following types: 


\begin{itemize}
\item{{\em boolean constants} -- used mostly by the metaobject concept definitions
like \verb@std::meta::Named@, \verb@std::meta::ScopeMember@, etc. and operations
like \verb@std::meta::is_anonymous@.}

\item{{\em integral constants} -- for operations like \verb@std::meta::get_source_line@
or \verb@std::meta::get_size@ and \verb@std::meta::get_constant@.}

\item{{\em pointer constants} -- for operations like \verb@std::meta::get_pointer@.}

\item{{\em c-string literals} -- for operations like \verb@std::meta::get_base_name@.}

\item{{\em metaobject IDs} -- for operations like \verb@std::meta::get_scope@,
\verb@std::meta::get_base_classes@ or \verb@std::meta::get_element@.}
\end{itemize}


So for example the Named concept is implemented with the help of
the \verb@__metaobject_is_meta_named@ compiler built-in which returns a boolean
compile time constant (equivalent to the following pseudocode):

\begin{minted}{cpp}
constexpr bool __metaobject_is_meta_named(__metaobject_id moid);
\end{minted}

The concept as defined in the specification can be then implemented in the
following manner: 

\begin{minted}{cpp}
namespace std {
namespace meta {

template <typename T>
concept bool Object = is_metaobject<T>;

template <Object T>
concept bool Named = __metaobject_is_meta_named(__unwrap_id<T>);

} // namespace meta
} // namespace std
\end{minted}

The \verb@get_base_name@ operation is implemented with the help of the
\verb@__metaobject_get_base_name@ built-in resulting in a \verb@constexpr@
\verb@char@ array: 

\begin{minted}{cpp}
constexpr const char [] __metaobject_get_base_name(__metaobject_id moid);

namespace std {
namespace meta {

template <Named T>
struct get_base_name
{
	constexpr const char value[] =
		__metaobject_get_base_name(__unwrap_id_v<T>);
};

} // namespace meta
} // namespace std
\end{minted}

There is also the \verb@std::meta::get_reflected_type@ operation which instead
of a compile-time constant value, yields a type. This operation is implemented
by the \verb@__unrefltype@ built-in operator:

\begin{minted}{cpp}
unspecified-type __unrefltype(__metaobject_id);
\end{minted}

The template \verb@get_reflected_type@ is then implemented as follows:

\begin{minted}{cpp}
namespace std {
namespace meta {

template <Type T>
struct get_reflected_type
{
	using type = __unrefltype(__unwrap_id_v<T>);
};

} // namespace meta
} // namespace std
\end{minted}
