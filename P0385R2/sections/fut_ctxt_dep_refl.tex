\subsection{Context-dependent reflection}
\label{context-dependent-reflection}

One of the interesting potential future extensions of reflection
is {\em context-dependent reflection}. Currently the operands of \verb@$reflect@
are names of the base-level entities which we want to reflect.

Context-dependent reflection would allow to obtain metadata based on the context
in which the reflection operator is invoked, instead of on the declaration name.

Several new expressions would have to be added as valid operands for
\verb@$reflect@, allowing to reflect the \say{current} namespace, class
or even function.

\subsubsection{Namespaces}

If the \verb@this::namespace@ expression was used as the operand of the reflection
operator, then it would return a \meta{Namespace} reflecting the innermost namespace
inside of which the reflection operator was invoked.

For example:

\begin{minted}{cpp}
typedef $reflect(this::namespace) _meta_gs;
\end{minted}

reflects the global scope namespace and is equivalent to

\begin{minted}{cpp}
typedef $reflect(::) _meta_gs;
\end{minted}

For named namespaces:

\begin{minted}{cpp}
namespace foo {
\end{minted}
\begin{minted}[xleftmargin=1em]{cpp}
typedef $reflect(this::namespace) _meta_foo;

namespace bar {
\end{minted}
\begin{minted}[xleftmargin=2em]{cpp}
typedef $reflect(this::namespace) _meta_foo_bar;

} // namespace bar
\end{minted}
\begin{minted}{cpp}
} // namespace foo
\end{minted}

\subsubsection{Classes}

If the \verb@this::class@ expression was used as the argument of the reflection
operator, then it would return a \meta{Class} reflecting the class
inside of which the reflection operator was invoked.

For example:

\begin{minted}[tabsize=4]{cpp}
struct foo
{
	const char* _name;

	// reflects foo
	typedef $reflect(this::class) _meta_foo1;

	foo(void)
	 : _name(get_base_name_v<$reflect(this::class)>())
	{ }

	void f(void)
	{
		// reflects foo
		typedef $reflect(this::class) _meta_foo2;
	}

	double g(double, double);

	struct bar
	{
		// reflects foo::bar
		typedef $reflect(this::class) _meta_foo_bar;
	};
};

double foo::g(double a, double b)
{
	// reflects foo
	typedef $reflect(this::class) _meta_foo3;
	return a+b;
}

class baz
{
private:
	typedef $reflect(this::class) _meta_baz;
};

typedef $reflect(this::class); // error: not used inside of a class.

\end{minted}

\subsubsection{Functions}

If the \verb@this::function@ expression was used as the argument of the reflection
operator, then it would return a \meta{Function} reflecting the function or operator
inside of which the reflection operator was invoked.

For example:

\begin{minted}[tabsize=4]{cpp}
void foobar(void)
{
	// reflects this particular overload of the foobar function
	typedef $reflect(this::function) _meta_foobar;
}

int foobar(int i)
{
	// reflects this particular overload of the foobar function
	typedef $reflect(this::function) _meta_foobar;
	return i+1;
}

class foo
{
private:
	void f(void)
	{
		// reflects this particular overload of foo::f
		typedef $reflect(this::function) _meta_foo_f;
	}

	double f(void)
	{
		// reflects this particular overload of foo::f
		typedef $reflect(this::function) _meta_foo_f;
		return 12345.6789;
	}
public:
	foo(void)
	{
		// reflects this constructor of foo
		typedef $reflect(this::function) _meta_foo_foo;
	}

	friend bool operator == (foo, foo)
	{
		// reflects this operator
		typedef $reflect(this::function) _meta_foo_eq;
	}

	typedef $reflect(this::function) _meta_fn; // error
};

typedef $reflect(this::function) _meta_fn; // error
\end{minted}

\subsubsection{Templates}

If the \verb@this::template@ expression was used as the argument of the reflection
operator, then it would return a \meta{Template} reflecting the class (or function)
template inside of which the reflection operator was invoked.

For example:

\begin{minted}[tabsize=4]{cpp}

template <typename First, typename Second, typename Third>
struct triplet
{
	// reflects the template not the instantiation
	using _meta_tpl = $reflect(this::template);

	// reflects the instantiation not the template
	using _meta_cls = $reflect(this::class);
};
\end{minted}
