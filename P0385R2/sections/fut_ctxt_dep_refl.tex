\subsection{Context-dependent reflection}
\label{context-dependent-reflection}

One of the interesting potential future extensions of reflection
is {\em context-dependent reflection}. Currently the operands of \verb@$reflect@
are names of the base-level entities which we want to reflect.

Context-dependent reflection would allow to obtain metadata based on the context
in which the reflection operator is invoked, instead of on the declaration name.

This could take the form of either something simple like:

\begin{minted}{cpp}
$reflect()
\end{minted}

which would reflect the current scope or even the current position in the
source code, or several new expressions could be added as valid operands for
\verb@$reflect@, allowing to reflect the current namespace, class, template
or function as described below.

\subsubsection{Namespaces}

If the \verb@this::namespace@ expression was used as the operand of the reflection
operator, then it would return a \meta{Namespace} reflecting the innermost namespace
inside of which the reflection operator was invoked.

For example:

\begin{minted}{cpp}
using _meta_gs = $reflect(this::namespace);
\end{minted}

reflects the global scope namespace and is equivalent to

\begin{minted}{cpp}
using _meta_gs = $reflect(::);
\end{minted}

For named namespaces:

\begin{minted}{cpp}
namespace foo {
\end{minted}
\begin{minted}[xleftmargin=1em]{cpp}
using _meta_foo = $reflect(this::namespace);

namespace bar {
\end{minted}
\begin{minted}[xleftmargin=2em]{cpp}
using _meta_foo_bar = $reflect(this::namespace);
\end{minted}
\begin{minted}[xleftmargin=1em]{cpp}
} // namespace bar
\end{minted}
\begin{minted}{cpp}
} // namespace foo
\end{minted}

\subsubsection{Classes}

If the \verb@this::class@ expression was used as the argument of the reflection
operator, then it would return a \meta{Class} reflecting the class
inside of which the reflection operator was invoked.

For example:

\begin{minted}[tabsize=4]{cpp}
struct foo {
	const char* _name;

	// reflects foo
	using _meta_foo1 = $reflect(this::class);

	foo(void)
	 : _name(get_base_name_v<$reflect(this::class)>())
	{ }

	void f(void) {
		// reflects foo
		using _meta_foo2 = $reflect(this::class);
	}

	double g(double, double);

	struct bar {
		// reflects foo::bar
		using _meta_foo_bar = $reflect(this::class);
	};
};

double foo::g(double a, double b) {
	// reflects foo
	using _meta_foo3 = $reflect(this::class);
	return a+b;
}

class baz {
private:
	using _meta_baz = $reflect(this::class);
};

using _meta_cls = $reflect(this::class); // error: not used inside of a class.

\end{minted}

\subsubsection{Functions}

If the \verb@this::function@ expression was used as the argument of the reflection
operator, then it would return a \meta{Function} reflecting the function or operator
inside of which the reflection operator was invoked.

For example:

\begin{minted}[tabsize=4]{cpp}
void foobar(void) {
	// reflects this particular overload of the foobar function
	using _meta_foobar = $reflect(this::function);
}

int foobar(int i) {
	// reflects this particular overload of the foobar function
	using _meta_foobar = $reflect(this::function);
	return i+1;
}

class foo {
private:
	void f(void) {
		// reflects this particular overload of foo::f
		using _meta_foo_f = $reflect(this::function);
	}

	double f(void) {
		// reflects this particular overload of foo::f
		using _meta_foo_f = $reflect(this::function);
		return 12345.6789;
	}
public:
	foo(void) {
		// reflects this constructor of foo
		using _meta_foo_foo = $reflect(this::function);
	}

	friend bool operator == (foo, foo) {
		// reflects this operator
		using _meta_foo_eq = $reflect(this::function);
	}

	using _meta_fn = $reflect(this::function); // error
};

using _meta_fn = $reflect(this::function); // error
\end{minted}

\subsubsection{Templates}

If the \verb@this::template@ expression was used as the argument of the reflection
operator, then it would return a \meta{Template} reflecting the class (or function)
template inside of which the reflection operator was invoked.

For example:

\begin{minted}[tabsize=4]{cpp}

template <typename First, typename Second, typename Third>
struct triplet {
	// reflects the template not the instantiation
	using _meta_tpl = $reflect(this::template);

	// reflects the instantiation not the template
	using _meta_cls = $reflect(this::class);
};
\end{minted}
