\subsection{Metaobject unique ID}
\label{fut-metaobject-uid}

In order to accommodate different programming paradigms, it may be advantageous
to represent metaobjects as both types {\em and} optionally also as
\verb@constexpr@ {\em values} and have the ability to convert between these
representations.

The compile-time value representation of a metaobject could be defined as:

\begin{minted}[tabsize=4]{cpp}
namespace reflect {
\end{minted}
\begin{minted}[xleftmargin=1em,tabsize=4]{cpp}
class object_uid {
public:
	constexpr object_uid(void) noexcept;
	constexpr object_uid(const object_uid&) noexcept;

	friend constexpr
	bool operator == (object_uid, object_uid) noexcept;

	friend constexpr
	bool operator != (object_uid, object_uid) noexcept;

	friend constexpr
	bool operator <  (object_uid, object_uid) noexcept;
};
\end{minted}
\begin{minted}[tabsize=4]{cpp}
} // namespace reflect
\end{minted}

The actual identifier could internally be something like a constant value of
\verb@int@, \verb@std::size_t@ or \verb@std::intptr_t@ storing an index, a hash
or an encoded (and possibly encrypted) pointer to the internal representation
of the metaobject in the compiler or something similar.
This would however be a hidden implementation detail.

The advantage of the \verb@object_uid@ is that we could represent heterogeneous
metaobjects -- different types, as \verb@constexpr@ instances of a homogeneous
type and store them in \verb@constexpr@ data structures and use them with
\verb@constexpr@ functions.

Instances of \verb@object_uid@ could be queried from a metaobject by a new
operation, implemented by the following template:

\begin{minted}[tabsize=4]{cpp}
namespace reflect {
\end{minted}
\begin{minted}[xleftmargin=1em,tabsize=4]{cpp}
template <Object T>
struct get_uid {
	constexpr const object_uid value = /* ... */;
};

template <Object T>
constexpr object_uid get_uid_v = get_uid<T>::value;
\end{minted}
\begin{minted}[tabsize=4]{cpp}
} // namespace reflect
\end{minted}

As long as an instance of \verb@reflect::object_uid@ stayed \verb@constexpr@, it would
also be possible to convert it back to a metaobject (type), for example by using
a new version of the reflection operator \verb@$reflect@:

\begin{minted}[tabsize=4]{cpp}
// pseudocode
constexpr Object $reflect(reflect::object_uid);
\end{minted}

This feature would make the use of reflection more convenient in some use cases
or under some paradigms and could help to simplify the implementation of the
higher-level fa\c{c}ades.

The experimental implementation described in section~\ref{implementation}
already represents metaobjects as integral constants and makes the addition
of the functionality described above trivial.

