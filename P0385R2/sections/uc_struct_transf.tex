\subsection{Structure data member transformations}
\label{use-case-struct-transf}

We need to create a new structure which has data members with the same or similar
names as an original structure, but we need to change some of the properties
of the data members, usually their types.

For example we need to transform a structure like:

\begin{minted}[tabsize=4]{cpp}
struct foo {
	bool b;
	char c;
	double d;
	float f;
	string s;
};
\end{minted}

into

\begin{minted}[tabsize=4]{cpp}
struct rdbs_table_placeholder_foo {
	column_placeholder_t<bool> b;
	column_placeholder_t<char> c;
	column_placeholder_t<double> d;
	column_placeholder_t<float> f;
	column_placeholder_t<string> s;
};
\end{minted}

or create a structure-of-arrays, which was one of the \say{targeted use cases}
from the committee's CFP~\cite{ISOCPP-N3814}: 

\begin{minted}[tabsize=4]{cpp}
struct soa_foo {
	vector<bool> bs;
	vector<char> cs;
	vector<double> sd;
	vector<float> fs;
	vector<string> ss;
};
\end{minted}

The primary obstacle to implementing this use case with the help of reflection
is at the moment the fact that we do not have the ability to create C++
identifiers \say{programmatically}, at least not without the help of
the preprocessor.

We either have to add the ability to create identifiers from compile-time strings
which may be fairly complicated, or look for some simpler workarounds.

First, let's say that we have a magic operator, like \verb@$identifier@ as
described in \ref{fut-op-identifier1} , which \say{creates} an identifier from
a compile-time string, so that for example:

\begin{minted}[tabsize=4]{cpp}
$identifier("long") $identifier("foo")(int i, int j);
\end{minted}

would be equivalent to 

\begin{minted}[tabsize=4]{cpp}
long foo(int i, int j);
\end{minted}

To copy a name of another declaration from a metaobject reflecting it,
we can use the \verb@get_base_name@ operation, which returns a \verb@constexpr@
array of \verb@char@s:

\begin{minted}[tabsize=4]{cpp}
struct bar {
	int foo;
};

using meta_bar_foo = $reflect(bar::foo);

long $identifier(reflect::get_base_name_v<meta_bar_foo>)(int i, int j);
\end{minted}

In order to do anything more complex than just
copying the identifiers of other declarations we would need a compile-time
string manipulation utility implementing for example functions for
compile-time string concatenation, like \verb@ct_concat@:

\begin{minted}[tabsize=4]{cpp}
int $identifier(ct_concat("get_", reflect::get_base_name_v<meta_bar_foo>))(const bar&);
\end{minted}

which would be equivalent to:

\begin{minted}[tabsize=4]{cpp}
int get_foo(const bar&);
\end{minted}

With the help of multiple inheritance and the
\hyperref[fac-unpack-sequence]{\texttt{unpack\_sequence}} helper template described
in section \ref{fac-unpack-sequence}, we can create a new structure that is
nearly equivalent to \verb@soa_foo@ via metaprogramming:

\begin{minted}[tabsize=4]{cpp}

template <typename MetaDataMember>
struct soa_single_member {
	// vector<T> Xs;
	vector<reflect::get_reflected_type_t<reflect::get_type_t<MetaDataMember>>>
		$identifier(ct_concat(reflect::get_base_name_v<MetaDataMember>, "s"));

	/* constructors, forwarding parameters to the vector, ... */
};

template <typename ... MetaDataMembers>
struct soa_inherit_all
 : soa_single_member<MetaDataMembers>...  {
 /* constructors forwarding parameters to the inherited soa_single_members */
 };

template <typename T>
struct soa
 : reflect::unpack_sequence_t<
	reflect::get_data_members_t<$reflect(T)>,
	soa_inherit_all
> {
/* constructors forwarding parameters to soa_inherit_all */
};

using soa_foo = soa<foo>;
\end{minted}

If we don't want to implement the operator \verb@$identifier@,
we could instead add a new template \verb@named_data_member@
for \meta{Named} metaobjects, described in greater detail in section
\ref{fut-named-member-X}:

\begin{minted}[tabsize=4]{cpp}
template <MetaNamed MO, typename X>
struct named_data_member;
\end{minted}

A specialization of \verb@named_data_member@ for a metaobject reflecting
for example \verb@foo::c@, would look like this:

\begin{minted}[tabsize=4]{cpp}
template <typename X>
struct named_data_member<$reflect(foo::c), X> {
	struct type {
		X c;

		template <typename ... P>
		type(P&& ... p)
		 : c(forward<P>(p)...) 
		{ }
	};
};
\end{minted}

We can then combine the specializations of \verb@named_data_member<...>::type@
via multiple inheritance as before:

\begin{minted}[tabsize=4]{cpp}

template <typename MetaDataMember>
using soa_single_member = reflect::named_data_member_t<
	MetaDataMember,
	vector<reflect::get_reflected_type_t<reflect::get_type_t<MetaDataMember>>>
>;

template <typename ... MetaDataMember>
struct soa_inherit_all
 : soa_single_member<MetaDataMembers>...  {
 /* constructors forwarding parameters to the inherited soa_single_members */
 };

template <typename T>
struct soa
 : reflect::unpack_sequence_t<
	reflect::get_data_members_t<$reflect(T)>,
	soa_inherit_all
> {
/* constructors forwarding parameters to soa_inherit_all */
};

using soa_foo = soa<foo>;
\end{minted}

Note that with this workaround we don't have the ability to change the name
of the data members from \verb@c@ to \verb@cs@, etc., but the \verb@named_data_member@
template is potentially much simpler to implement.

For a more detailed discussion on the possibilities of identifier generation
see section \ref{fut-ident-gen} and also section \ref{fut-variadic-composition}.
