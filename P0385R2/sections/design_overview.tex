\subsection{Basic overview}

As the introduction briefly mentions, the metadata reflecting base-level
program declarations has its own structure. One way to maintain this structure
and to organize the individual, but related pieces of metadata reflecting
for example the structure of a class is to compose them into {\em metaobjects}.

In P0194 we propose to add support for
compile-time reflection to C++ by the means of lightweight, compiler-generated
types -- {\em metaobjects}, providing \hyperref[term-metadata]{metadata}
describing various \hyperref[term-base-meta-level]{base-level} program declarations.

The Metaobjects themselves are opaque, without any visible internal structure.
Values of such a Metaobject type are default- and copy-constructible.

Their primary purpose is to give a \hyperref[term-first-class]{first-class identity}
to the reflected entity\footnote{Namespace, type alias, function, parameter,
specifier, etc.}, so that we can pass it as an argument or a return value
in metaprograms and to separate the reflection of a declaration from the querying
of metadata\footnote {Which will happen very often in the more complex use cases}.

We introduce a new reflection operator -- \verb@$reflect@ which returns a
metaobject type reflecting its operand.

For example:

\begin{minted}{cpp}
using meta_global_scope     = $reflect(::);                 // GlobalScope
using meta_int              = $reflect(int);                // Type
using meta_std              = $reflect(std);                // Namespace
using meta_std_size_t       = $reflect(std::size_t);        // Type and Alias
using meta_std_thread       = $reflect(std::thread);        // Class
using meta_std_pair         = $reflect(std::pair<int, int>);// Class
using meta_std_launch       = $reflect(std::launch);        // Enum
using meta_std_launch_async = $reflect(std::launch::async); // Enumerator
\end{minted}

or in the future revisions;

\begin{minted}{cpp}
using meta_std_pair         = $reflect(std::pair);          // Template
using meta_std_sin          = $reflect(std::sin);           // OverloadedFunction
\end{minted}

Since there are many different kinds of
base-level declarations, the metaobjects reflecting them are
modeling various {\em metaobject concepts}, which also serve to classify
metaobjects and to indicate whether a metaobjects has or has not a particular
property;

\begin{minted}{cpp}
static_assert(reflect::Named<$reflect(std)>);
static_assert(reflect::ScopeMember<$reflect(int)>);
static_assert(reflect::Scope<$reflect(std::string)>);
static_assert(reflect::Alias<$reflect(std::string::size_type)>);
\end{minted}

or if is falls into a particular category:

\begin{minted}{cpp}
static_assert(reflect::GlobalScope<$reflect()>);
static_assert(reflect::Namespace<$reflect(std)>);
static_assert(reflect::Type<$reflect(int)>);
static_assert(reflect::Class<$reflect(std::string)>);
static_assert(reflect::Specifier<$reflect(virtual)>);
\end{minted}

The individual pieces of metadata can be obtained from a metaobject by using one
of the class templates which comprise its interface.

Some of this metadata like the class name or number of base classes is provided
as compile-time constant values, some as base-level types
and some in the form of other metaobjects, like metaobjects
reflecting declaration scope or a sequence of metaobjects reflecting class members,
etc.:

\begin{minted}{cpp}
using meta_str = $reflect(std::string);

get_base_name_v<meta_str>; // a compile-time constant string
get_reflected_type_t<meta_str>; // a base-level type: std::string
get_scope_t<meta_str>; // another metaobject reflecting the scope
get_data_members_t<meta_str>; // a metaobject-sequence containing other metaobjects
\end{minted}

P0194R3 also defines the initial subset
of metaobject concepts which we assume to be essential
and which will provide a good starting point for future extensions.

