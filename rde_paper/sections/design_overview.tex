\subsection{Basic overview}

In P0194R0 (and its predecessors) we propose to add native support for
compile-time reflection to C++ by the means of lightweight compiler-generated
types -- {\em metaobjects} providing metadata describing various program
declarations.

The metaobjects exist only at the type-level, they do not have any constructors
or members and cannot be instantiated. Their sole purpose is to give an identity
to the reflected entity (namespace, class, function, variable, specifier, etc.).

The metadata can be obtained by using one of the class templates which comprise
the interface of the metaobjects. Since there are many different kinds of
base-level reflectible declarations, the metaobjects reflecting them are
modeling various {\em metaobject concepts}. The metaobjects can be inspected
by {\em metaobject traits}, which indicate whether a metaobjects has or has
not a particular property or if is falls into a particular category.

We also introduce a new reflection operator -- \verb@reflexpr@ and the initial subset
of metaobject concepts which we assume to be essential
and which will provide a good starting point for future extensions.

When finalized, these metaobjects, together with some additions to the standard
library can later be used to implement other third-party libraries
providing both compile-time and run-time, high-level reflection.

