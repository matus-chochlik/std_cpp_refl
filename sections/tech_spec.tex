\section{Technical Specifications}

In the proposed solution, the basic metadata describing a program written
in C++ would be made available through a set of anonymous types (classes)
defined by the compiler. These classes would describe various program
constructs like, namespaces, types, typedefs, classes, their member variables
(member data), member functions, inheritance, templates, template parameters,
enumerated values, etc.

Since we want the metadata to be static, different base-level constructs
would be described by {\em "statically" different} metaobjects and this
by {\em different} types.
For example a metaobject describing the global scope namespace would
be a different {\em type} than a metaobject (type) describing the \verb@std@
namespace, etc.

This way special types (metaobjects) become "instances" of meta-level
concepts (static interfaces which would not exist as concrete types,
but rather only at the "documentation-level").
This section describes a hierarchy of these concepts 
for metaobjects, tag types for metaobject classification and functions (or operators)
providing access to the metaobjects.

\subsection{}

\subsection{Specifier Categorization}

\subsection{Basic Metaobject Categorization}

\subsection{Specifier Concepts}

\subsection{Basic Metaobject Concepts}

The concepts described here would not exist as concrete types and would not
be defined by the standard.
They only describe which member typedefs, member functions, etc. are available
for various kinds of metaobjects.
